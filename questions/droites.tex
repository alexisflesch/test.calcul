\begin{qcm}{droites}
    \begin{question}{droites.1.n}
        Dans le plan muni d'un repère $(O;\overrightarrow{i},\overrightarrow{j})$, la droite $\Delta$ d'équation $x+2y+3=0$ coupe l'axe des abscisses au point $A$ d'abscisse:
         \vspace{-1.5ex}
         \begin{multicols}{4}
         \begin{reponses}     
            \bonne{$ -3$}
            \mauvaise{$-3/2$}
            \mauvaise{$3$}
            \mauvaise{$ 0$}
         \end{reponses}
         \end{multicols}
    \end{question}
\end{qcm}


\begin{qcm}{droites}
    \begin{question}{droites.2.n}
        Dans le plan muni d'un repère $(O;\overrightarrow{i},\overrightarrow{j})$, la droite $\Delta$ d'équation $x+2y+3=0$ coupe l'axe des ordonnées au point $A$ d'ordonnée:
         \vspace{-1.5ex}
         \begin{multicols}{4}
         \begin{reponses}         
            \bonne{$-3/2$}
            \mauvaise{$-3$}
            \mauvaise{$3$}
            \mauvaise{$ 0$}
         \end{reponses}
         \end{multicols}
    \end{question}
\end{qcm}


\begin{qcm}{droites}
    \begin{question}{droites.3.n}
        Dans le plan muni d'un repère $(O;\overrightarrow{i},\overrightarrow{j})$, la droite $\Delta$ d'équation $2x+y+3=0$ coupe l'axe des abscisses au point $A$ d'abscisse:
         \vspace{-1.5ex}
         \begin{multicols}{4}
         \begin{reponses}       
            \bonne{$ -3/2$}
            \mauvaise{$-3$}
            \mauvaise{$3$}
            \mauvaise{$ 0$}
         \end{reponses}
         \end{multicols}
    \end{question}
\end{qcm}


\begin{qcm}{droitesbis}
    \begin{question}{droitesbis.1.n}
        Dans un repère $(O;\overrightarrow{i},\overrightarrow{j})$ du plan, on se donne
        les points $A(2;0)$ et $B(0;b)$ où $b$ est un réel non
        nul. Le point d'intersection de la droite $(AB)$ avec la droite
        d'équation $y=3$ a pour coordonnées:
        \vspace{-0.2cm}
        \begin{multicols}{2}
            \begin{reponses}       
                \bonne{$\left(2-\dfrac{6}{b};3\right) $}
                \mauvaise{$\left(2+\dfrac{6}{b};3\right) $}
                \mauvaise{$\left(1+\dfrac{3}{b};3\right) $}
                \mauvaise{$\left(1-\dfrac{3}{b};3\right) $}
                \mauvaise{$\left(-\dfrac{2}{b};3\right) $}
                \alafin
                \mauvaise{Aucune des réponses précédentes}        
            \end{reponses} 
        \end{multicols}
    \end{question}
\end{qcm}

\begin{qcm}{droitesbis}
    \begin{question}{droitesbis.2.n}
        Dans un repère $(O;\overrightarrow{i},\overrightarrow{j})$ du plan, on se donne
        les points $A(a;0)$ et $B(0;2)$ où $a$ est un réel non
        nul. Le point d'intersection de la droite $(AB)$ avec la droite
        d'équation $x=3$ a pour coordonnées:
        \vspace{-0.2cm}
        \begin{multicols}{2}
            \begin{reponses}       
                \bonne{$\left(3;2-\dfrac{6}{a}\right) $}
                \mauvaise{$\left(3;2+\dfrac{6}{a}\right) $}
                \mauvaise{$\left(3;1+\dfrac{3}{a}\right) $}
                \mauvaise{$\left(3;1-\dfrac{3}{a}\right) $}
                \mauvaise{$\left(3;-\dfrac{2}{a}\right) $}
                \alafin
                \mauvaise{Aucune des réponses précédentes}        
            \end{reponses} 
        \end{multicols}
    \end{question}
\end{qcm}
