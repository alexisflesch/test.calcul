\documentclass[a4paper,11pt]{article}
% \usepackage[twocolumn,columnsep=1cm]{geometry}

\usepackage[utf8]{inputenc}
\usepackage[T1]{fontenc}
\usepackage{environ}
\usepackage[french]{babel}
\usepackage[francais,nowatermark]{automultiplechoice}
\usepackage{amsmath}
\usepackage{amssymb}
\usepackage{amsfonts}
\usepackage{mathrsfs} 
\usepackage{stmaryrd}
\usepackage{multicol}
\usepackage{vwcol}
\usepackage{xcolor}
\usepackage{microtype}



\def\AMCchoiceLabel#1{}

\NewEnviron{qcm}[1]{%
  \def\arg{#1}%
  \expandafter\element\expandafter\arg\expandafter{\BODY}}

\makeatletter
\renewcommand{\element}[2]{%
  \AMC@prepare@element{#1}%
  \global\csname #1@\romannumeral\AMCtok@k\endcsname={#2}%
}
\makeatother

\AMCrandomseed{17}
\AMCboxDimensions{size=2ex,down=.2ex}
\AMCboxStyle {size=1.7ex,down=0ex,rule=.8pt}

\geometry{left=1.5cm,right=1.5cm}

\AMCinterIrep=.8ex

\def\AMCbeginQuestion#1#2{\par\noindent{\bf \fbox{#1}} #2}
\def\multiSymbole{}
\setlength{\parindent}{0pt}

\begin{document}

\AMCnumericOpts{scoreexact=1}
\baremeDefautS{e=-1,v=0,b=1,m=-1}


\setlength\columnsep{1cm}

\begin{qcm}{developpement.factorisation.developpement}
  \begin{question}{developpement.factorisation.dl.1.a}
    Développer \((2x-4)^2\).
    \vspace{-1.5ex}
    \begin{multicols}{2}
      \begin{reponses}
        \bonne{\(4x^2-16x+16\)}
        \mauvaise{\(2x^2-16x+16\)}
        \mauvaise{\(4x^2-16x-16\)}
        \mauvaise{\(2x^2+16x-16\)}
        \mauvaise{\(4x^2-8x+16\)}
      \end{reponses}
      \setcounter{unbalance}{1}
    \end{multicols}
  \end{question}
\end{qcm}

\begin{qcm}{developpement.factorisation.developpement}
  \begin{question}{developpement.factorisation.dl.2.a}
    Développer \((3x-2)^2\).
    \vspace{-1.5ex}
    \begin{multicols}{2}
      \begin{reponses}
        \bonne{\(9x^2-12x+4\)}
        \mauvaise{\(9x^2-6x+4\)}
        \mauvaise{\(3x^2-12x+4\)}
        \mauvaise{\(9x^2-12x-4\)}
        \mauvaise{\(3x^2+12x-4\)}
      \end{reponses}
      \setcounter{unbalance}{1}
    \end{multicols}
  \end{question}
\end{qcm}



%%%%%%%%%%%%%%%%%%%%%%%%%

\begin{qcm}{developpement.factorisation.factorisation}
  \begin{question}{developpement.factorisation.fact.1.a}
    Factoriser \(25x^2-36\).
    \vspace{-1.5ex}
    \begin{multicols}{2}
      \begin{reponses}
        \bonne{\((5x-6)(5x+6)\)}
        \mauvaise{\((25x-6)(25x+6)\)}
        \mauvaise{\((5x-6)^2\)}
        \mauvaise{\((5x+6)^2\)}
      \end{reponses}
    \end{multicols}
  \end{question}
\end{qcm}


\begin{qcm}{developpement.factorisation.factorisation}
  \begin{question}{developpement.factorisation.fact.2.a}
    Factoriser \(4x^2-9\).
    \vspace{-1.5ex}
    \begin{multicols}{2}
      \begin{reponses}
        \bonne{\((2x-3)(2x+3)\)}
        \mauvaise{\((4x-3)(4x+2)\)}
        \mauvaise{\((2x-3)^2\)}
        \mauvaise{\((2x+3)^2\)}
      \end{reponses}
    \end{multicols}
  \end{question}
\end{qcm}

\begin{qcm}{developpement.factorisation.puissances}
    \begin{question}{developpement.factorisation.puissances.1.k}
         Simplifier \(\dfrac{xy+y^2}{x^4+2x^3y+x^2y^2}\) où \(x\neq0\) et \(y\neq0\).
         \vspace{-1.5ex}
         \begin{multicols}{3}
         \begin{reponses}
             \lastchoices
              \mauvaise{\(\dfrac{\sqrt{x+y}\ y}{x^2}\)}
              \mauvaise{\(\dfrac{1+x^3}{x^3}\)}
              \bonne{\(\dfrac{y}{x^2(x+y)}\)}
              \mauvaise{\(\dfrac{(x+y)y}{x}\)}
              \mauvaise{\(\dfrac{y}{x^2}\)}
              
              \phantom{foo}
         \end{reponses}
        \end{multicols}
    \end{question}
\end{qcm}

\begin{qcm}{developpement.factorisation.puissances}
    \begin{question}{developpement.factorisation.puissances.2.k}
    Simplifier \(\dfrac{x^2y^3}{x^5+2x^4y+x^3y^2}\) où \(x\neq0\) et \(y\neq0\).
         \vspace{-1.5ex}
         \begin{multicols}{3}
         \begin{reponses}
             \lastchoices
              \mauvaise{\(\dfrac{y^3}{x^3}\)}
              \mauvaise{\(\dfrac{x^{\frac{2}{3}}y^3}{(x+y)^2}\)}
              \mauvaise{\(\dfrac{y}{x^2}\)}
              \mauvaise{\(\dfrac{1}{x^3}\)}
              \bonne{\(\dfrac{y^3}{(x+y)^2 x}\)}
              
              \phantom{foo}
         \end{reponses}
        \end{multicols}
    \end{question}
\end{qcm}


\begin{qcm}{expln.simple}
    \begin{question}{expln.simple.1.a}
         \(\mathrm{e}^{2\ln(3)}=\ldots\)
         \vspace{-1.5ex}
         \begin{multicols}{3}
         \begin{reponses}
            \bonne{\(9 \textcolor{white}{\dfrac{3}{2}}\)}
            \mauvaise{\(\dfrac{3}{2}\)}
            \mauvaise{\(5 \textcolor{white}{\dfrac{3}{2}}\)}
            \mauvaise{\(6\textcolor{white}{\dfrac{3}{2}}\)}
            \mauvaise{\(\dfrac{2}{3}\)}
            \mauvaise{\(4\textcolor{white}{\dfrac{3}{2}}\)}
         \end{reponses}
         \end{multicols}
    \end{question}
\end{qcm}

\begin{qcm}{expln.simple}
    \begin{question}{expln.simple.2.a}
         \(\mathrm{e}^{3\ln(2)}=\ldots\)
         \vspace{-1.5ex}
         \begin{multicols}{3}
         \begin{reponses}
            \bonne{\(8 \textcolor{white}{\dfrac{3}{2}}\)}
            \mauvaise{\(\dfrac{3}{2}\)}
            \mauvaise{\(4 \textcolor{white}{\dfrac{3}{2}}\)}
            \mauvaise{\(6 \textcolor{white}{\dfrac{3}{2}}\)}
            \mauvaise{\(5 \textcolor{white}{\dfrac{3}{2}}\)}
            \mauvaise{\(\dfrac{2}{3}\)}
         \end{reponses}
         \end{multicols}
    \end{question}
\end{qcm}

\begin{qcm}{expln.simple}
    \begin{question}{expln.simple.3.a}
         \(\mathrm{e}^{2\ln(5)}=\ldots\)
         \vspace{-1.5ex}
         \begin{multicols}{3}
         \begin{reponses}
            \mauvaise{\(32 \textcolor{white}{\dfrac{3}{2}}\)}
            \mauvaise{\(10 \textcolor{white}{\dfrac{3}{2}}\)}
            \bonne{\(25 \textcolor{white}{\dfrac{3}{2}}\)}            
            \mauvaise{\(\dfrac{5}{2}\)}
            \mauvaise{\(\dfrac{2}{5}\)}
            \mauvaise{\(4 \textcolor{white}{\dfrac{3}{2}}\)}
         \end{reponses}
         \end{multicols}
    \end{question}
\end{qcm}

\begin{qcm}{expln.simple}
    \begin{question}{expln.simple.4.a}
         \(\mathrm{e}^{5\ln(2)}=\ldots\)
         \vspace{-1.5ex}
         \begin{multicols}{3}
         \begin{reponses}
            \mauvaise{\(25 \textcolor{white}{\dfrac{3}{2}}\)}
            \mauvaise{\(10 \textcolor{white}{\dfrac{3}{2}}\)}
            \mauvaise{\(\dfrac{5}{2}\)}
            \bonne{\(32 \textcolor{white}{\dfrac{3}{2}}\)}
            \mauvaise{\(\dfrac{2}{5}\)}
            \mauvaise{\(4 \textcolor{white}{\dfrac{3}{2}}\)}
         \end{reponses}
         \end{multicols}
    \end{question}
\end{qcm}


%%%%%%%%%%%%%%%%%%%%%%%

\begin{qcm}{expln.moyen}
    \begin{question}{expln.moyen.1.a}
         Soient \(x\) et \(y\) deux réels. Alors, \(\mathrm{e}^{xy}=\ldots\)
         \vspace{-1.5ex}
         \begin{multicols}{2}
         \begin{reponses}
            \mauvaise{\(\mathrm{e}^x\mathrm{e}^y\)}
            \mauvaise{\(\mathrm{e}^x+\mathrm{e}^y\)}
            \mauvaise{\(\dfrac{\mathrm{e}^x}{\mathrm{e}^y}\)}
            \mauvaise{\(\dfrac{\mathrm{e}^y}{\mathrm{e}^x}\)}
            \lastchoices
            \bonne{Aucune des réponses précédentes}
         \end{reponses}
         \end{multicols}
    \end{question}
\end{qcm}


\begin{qcm}{expln.moyen}
    \begin{question}{expln.moyen.2.a}
         Soient \(x\) et \(y\) deux réels. Alors, \(\mathrm{e}^{x+y}=\ldots\)
         \vspace{-1.5ex}
         \begin{multicols}{2}
         \begin{reponses}
            \bonne{\(\mathrm{e}^x\mathrm{e}^y\)}
            \mauvaise{\(\mathrm{e}^x+\mathrm{e}^y\)}
            \mauvaise{\(\dfrac{\mathrm{e}^x}{\mathrm{e}^y}\)}
            \mauvaise{\(\dfrac{\mathrm{e}^y}{\mathrm{e}^x}\)}
            \lastchoices
            \mauvaise{Aucune des réponses précédentes}
         \end{reponses}
         \end{multicols}
    \end{question}
\end{qcm}


\begin{qcm}{expln.moyen}
    \begin{question}{expln.moyen.3.a}
         Soient \(x>0\) et \(y>0\). Alors, \(\ln(x+y)=\ldots\)
         \vspace{-1.5ex}
         \begin{multicols}{2}
         \begin{reponses}
            \mauvaise{\(\ln(x)\ln(y)\)}
            \mauvaise{\(\ln(x)+\ln(y)\)}
            \mauvaise{\(\dfrac{\ln(x)}{\ln(y)}\)}
            \mauvaise{\(\dfrac{\ln(y)}{\ln(x)}\)}
            \lastchoices
            \bonne{Aucune des réponses précédentes}
         \end{reponses}
         \end{multicols}
    \end{question}
\end{qcm}

\begin{qcm}{expln.moyen}
    \begin{question}{expln.moyen.4.a}
         Soient \(x>0\) et \(y>0\). Alors, \(\ln(xy)=\ldots\)
         \vspace{-1.5ex}
         \begin{multicols}{2}
         \begin{reponses}
            \bonne{\(\ln(x)+\ln(y)\)}
            \mauvaise{\(\ln(x)\ln(y)\)}
            \mauvaise{\(\dfrac{\ln(x)}{\ln(y)}\)}
            \mauvaise{\(\dfrac{\ln(y)}{\ln(x)}\)}
            \lastchoices
            \mauvaise{Aucune des réponses précédentes}
         \end{reponses}
         \end{multicols}
    \end{question}
\end{qcm}



%%%%%%%%%%%%%%%%%%%%%%%

\begin{qcm}{expln.dur}
    \begin{question}{expln.dur.1.a}
         Soient \(x>0\) et \(y>0\). Alors, \(\ln\left(x\mathrm{e}^y\right)=\ldots\)
         \vspace{-1.5ex}
         \begin{multicols}{3}
         \begin{reponses}
            \bonne{\(\ln(x)+y\)}
            \mauvaise{\(\ln(y)+x\)}
            \mauvaise{\(x^y\)}
            \mauvaise{\(y^x\)}
            \mauvaise{\(\mathrm{e}^{xy}\)}
            \mauvaise{\(xy\)}
         \end{reponses}
         \end{multicols}
    \end{question}
\end{qcm}


\begin{qcm}{expln.dur}
    \begin{question}{expln.dur.2.a}
         Soient \(x>0\) et \(y>0\). Alors, \(\ln\left(y\mathrm{e}^x\right)=\ldots\)
         \vspace{-1.5ex}
         \begin{multicols}{3}
         \begin{reponses}
            \mauvaise{\(\ln(x)+y\)}
            \bonne{\(\ln(y)+x\)}
            \mauvaise{\(x^y\)}
            \mauvaise{\(y^x\)}
            \mauvaise{\(\mathrm{e}^{xy}\)}
            \mauvaise{\(xy\)}
         \end{reponses}
         \end{multicols}
    \end{question}
\end{qcm}



\begin{qcm}{racine.simple}
    \begin{question}{racine.simple.1.a}
         Simplifier \(\sqrt{20}\).
         \vspace{-1.5ex}
         \begin{multicols}{3}
         \begin{reponses}
            \bonne{\(2\sqrt{5}\)}
            \mauvaise{\(5\sqrt{2}\)}
            \mauvaise{\(\sqrt{2}+\sqrt{5}\)}
            \mauvaise{\(5+\sqrt{2}\)}
            \mauvaise{\(2+\sqrt{5}\)}
         \end{reponses}
         \end{multicols}
    \end{question}
\end{qcm}

\begin{qcm}{racine.simple}
    \begin{question}{racine.simple.2.a}
         Simplifier \(\sqrt{12}\).
         \vspace{-1.5ex}
         \begin{multicols}{3}
         \begin{reponses}
            \bonne{\(2\sqrt{3}\)}
            \mauvaise{\(3\sqrt{2}\)}
            \mauvaise{\(\sqrt{2}+\sqrt{3}\)}
            \mauvaise{\(3+\sqrt{2}\)}
            \mauvaise{\(2+\sqrt{3}\)}
         \end{reponses}
         \end{multicols}
    \end{question}
\end{qcm}

\begin{qcm}{racine.simple}
    \begin{question}{racine.simple.3.a}
         Simplifier \(\sqrt{18}\).
         \vspace{-1.5ex}
         \begin{multicols}{3}
         \begin{reponses}
            \bonne{\(3\sqrt{2}\)}
            \mauvaise{\(2\sqrt{3}\)}
            \mauvaise{\(\sqrt{2}+\sqrt{3}\)}
            \mauvaise{\(3+\sqrt{2}\)}
            \mauvaise{\(2+\sqrt{3}\)}
         \end{reponses}
         \end{multicols}
    \end{question}
\end{qcm}



%%%%%%%%%%%%%%%%%%%%%%%%%

\begin{qcm}{racine.moyen}
    \begin{question}{racine.moyen.1.a}
      Soient \(x\) et \(y\) des réels tels que \(x^2-y^2\geq 0\). Alors, \(\sqrt{x^2-y^2}=\ldots\)
         \vspace{-1.5ex}
         \begin{multicols}{2}
         \begin{reponses}
            \mauvaise{\(x-y\)}
            \mauvaise{\(|x|-|y|\)}
            \mauvaise{\(|x|+|y|\)}
            \lastchoices
            \bonne{Aucune des réponses précédentes}
         \end{reponses}
         \end{multicols}
    \end{question}
\end{qcm}


\begin{qcm}{racine.moyen}
    \begin{question}{racine.moyen.2.a}
     Soient \(x\) et \(y\) deux réels. Alors, \(\sqrt{x^2+y^2}=\ldots\)
         \vspace{-1.5ex}
         \begin{multicols}{2}
         \begin{reponses}
            \mauvaise{\(x+y\)}
            \mauvaise{\(|x|-|y|\)}
            \mauvaise{\(|x|+|y|\)}
            \lastchoices
            \bonne{Aucune des réponses précédentes}
         \end{reponses}
         \end{multicols}
    \end{question}
\end{qcm}


\begin{qcm}{signe.polynome}
    \begin{question}{signe.polynome.1.a}
         Quel est l'ensemble des solutions réelles de \(x^2+x-2\geqslant 0\)?
         \vspace{0.1cm}
         \begin{reponses}
            \bonne{\(]-\infty,-2]\cup[1,+\infty[\)}
            \mauvaise{\(]-\infty,-1]\cup[2,+\infty[\)}
            \mauvaise{\([0,1]\)}
            \mauvaise{\([-2,1]\)}
            \mauvaise{\([-1,2]\)}
         \end{reponses}
         \vspace{0.4cm}
    \end{question}
\end{qcm}


\begin{qcm}{signe.polynome}
    \begin{question}{signe.polynome.2.a}
         Quel est l'ensemble des solutions réelles de \(x^2+x-2\leqslant 0\)?
         \vspace{0.1cm}
         \begin{reponses}
            \mauvaise{\(]-\infty,-2]\cup[1,+\infty[\)}
            \mauvaise{\(]-\infty,-1]\cup[2,+\infty[\)}
            \mauvaise{\([0,1]\)}
            \bonne{\([-2,1]\)}
            \mauvaise{\([-1,2]\)}
         \end{reponses}
         \vspace{0.4cm}
    \end{question}
\end{qcm}

\begin{qcm}{signe.polynome}
    \begin{question}{signe.polynome.3.a}
         Quel est l'ensemble des solutions réelles de \(x^2+3x-4\geqslant 0\)?
         \vspace{0.1cm}
         \begin{reponses}
            \bonne{\(]-\infty,-4]\cup[1,+\infty[\)}
            \mauvaise{\(]-\infty,-1]\cup[4,+\infty[\)}
            \mauvaise{\([0,1]\)}
            \mauvaise{\([-4,1]\)}
            \mauvaise{\([-1,4]\)}
         \end{reponses}
         \vspace{0.4cm}
    \end{question}
\end{qcm}


\begin{qcm}{signe.polynome}
    \begin{question}{signe.polynome.4.a}
         Quel est l'ensemble des solutions réelles de \(x^2+3x-4\leqslant 0\)?
         \vspace{0.1cm}
         \begin{reponses}
            \mauvaise{\(]-\infty,-4]\cup[1,+\infty[\)}
            \mauvaise{\(]-\infty,-1]\cup[4,+\infty[\)}
            \mauvaise{\([0,1]\)}
            \bonne{\([-4,1]\)}
            \mauvaise{\([-1,4]\)}
         \end{reponses}
         \vspace{0.4cm}
    \end{question}
\end{qcm}


\begin{qcm}{signe.polynome}
    \begin{question}{signe.polynome.5.a}
         Quel est l'ensemble des solutions réelles de \(x^2+2x-3\geqslant 0\)?
         \vspace{0.1cm}
         \begin{reponses}
            \bonne{\(]-\infty,-3]\cup[1,+\infty[\)}
            \mauvaise{\(]-\infty,-1]\cup[3,+\infty[\)}
            \mauvaise{\([0,1]\)}
            \mauvaise{\([-3,1]\)}
            \mauvaise{\([-1,3]\)}
         \end{reponses}
         \vspace{0.4cm}
    \end{question}
\end{qcm}


\begin{qcm}{signe.polynome}
    \begin{question}{signe.polynome.6.a}
         Quel est l'ensemble des solutions réelles de \(x^2+2x-3\leqslant 0\)?
         \vspace{0.1cm}
         \begin{reponses}
            \mauvaise{\(]-\infty,-3]\cup[1,+\infty[\)}
            \mauvaise{\(]-\infty,-1]\cup[3,+\infty[\)}
            \mauvaise{\([0,1]\)}
            \bonne{\([-3,1]\)}
            \mauvaise{\([-1,3]\)}
         \end{reponses}
         \vspace{0.4cm}
    \end{question}
\end{qcm}

\begin{qcm}{fractions.lettres.simple}
    \begin{question}{fractions.lettres.simple.1.l}
         Soit \(x\) un réel non nul. Alors \( \dfrac{1}{x}+\dfrac{x}{2}=\ldots\)
%          \vspace{-1.5ex}
         \begin{multicols}{2}
            \begin{reponses}
              \mauvaise{\( \dfrac{1}{x}+\dfrac{1}{2} \)}
              \mauvaise{\( \dfrac{2+x}{x} \)}
              \mauvaise{\( \dfrac{2+x}{2x} \)}
              \bonne{\( \dfrac{2+x^2}{2x} \)}
         \end{reponses}
         \end{multicols}
    \end{question}
\end{qcm}


\begin{qcm}{fractions.lettres.simple}
    \begin{question}{fractions.lettres.simple.2.a}
         Soit \(x\) un réel non nul. Alors \( \dfrac{3}{x}+\dfrac{x}{2}=\ldots\)
%          \vspace{-1.5ex}
         \begin{multicols}{2}
            \begin{reponses}
              \mauvaise{\( \dfrac{3}{x}+\dfrac{1}{2} \)}
              \mauvaise{\( \dfrac{3+x}{2x} \)}
              \mauvaise{\( \dfrac{3+x}{x+2} \)}
              \bonne{\( \dfrac{6+x^2}{2x} \)}
         \end{reponses}
         \end{multicols}
    \end{question}
\end{qcm}



\begin{qcm}{fractions.lettres}
\begin{question}{fractions.lettres.2.l}
      Soit \(x\) un réel non nul. Simplifier : \( \dfrac{\dfrac{3}{2x}}{~~x^2~~}\).
%          \vspace{-1.5ex}
         \begin{multicols}{3}
            \begin{reponses}
              \mauvaise{\( \dfrac{3x}{2} \)}
              \mauvaise{\( \dfrac{3}{2x} \)}
              \bonne{\( \dfrac{3}{2x^3} \)}
              \mauvaise{\(\dfrac{6}{x}\)}
              \mauvaise{\(\dfrac{2}{3x}\)}
         \end{reponses}
         \end{multicols}
    \end{question}
\end{qcm}

\begin{qcm}{fractions.lettres}
\begin{question}{fractions.lettres.3.l}
         Soit \(x\) un réel non nul. Simplifier : \( \dfrac{3x}{~~\dfrac{2}{x^2}~~}\).
%          \vspace{-1.5ex}
         \begin{multicols}{3}
            \begin{reponses}
              \mauvaise{\( \dfrac{3}{2x} \)}
              \mauvaise{\( \dfrac{3}{2x^3} \)}
              \bonne{\( \dfrac{3x^3}{2} \)}
              \mauvaise{\(\dfrac{2}{3x^3}\)}
              \mauvaise{\( \dfrac{6}{x} \)}
         \end{reponses}
         \setcounter{unbalance}{1}
         \end{multicols}
    \end{question}
\end{qcm}

\begin{qcm}{fractions.lettres}
\begin{question}{fractions.lettres.4.l}
         Soit \(x\) un réel non nul. Simplifier : \( \dfrac{3x}{~~\dfrac{x^2}{2}~~}\).
%          \vspace{-1.5ex}
         \begin{multicols}{3}
            \begin{reponses}
              \mauvaise{\( \dfrac{3}{2x} \)}
             \mauvaise{\( \dfrac{3x}{2} \)}
             \mauvaise{\( \dfrac{3}{2x^2} \)}
              \mauvaise{\( \dfrac{3}{2x^3} \)}
              \bonne{\( \dfrac{6}{x} \)}
         \end{reponses}
         \setcounter{unbalance}{1}
         \end{multicols}
    \end{question}
\end{qcm}


\begin{qcm}{fractions.lettres}
\begin{question}{fractions.lettres.5.l}
         Soit \(x\) un réel non nul. Simplifier : \( \dfrac{\dfrac{3}{2x}}{~~\dfrac{2}{x^2}~~}\).
%          \vspace{-1.5ex}
         \begin{multicols}{3}
            \begin{reponses}
              \mauvaise{\( \dfrac{3}{2x} \)}
              \mauvaise{\( \dfrac{3x}{2} \)}
              \mauvaise{\( \dfrac{3}{2x^2} \)}
              \mauvaise{\( \dfrac{3}{x^3} \)}
              \bonne{\( \dfrac{3x}{4} \)}
         \end{reponses}
         \setcounter{unbalance}{1}
         \end{multicols}
    \end{question}
\end{qcm}
\begin{qcm}{fractions.numeriques.priorites}
    \begin{question}{fractions.numeriques.priorites.1.l}
          Simplifier : \( \dfrac{2}{3}-\dfrac{1}{2}\times\dfrac{1}{4}=\ldots\) 
      \begin{multicols}{3}
         \begin{reponses}
              \mauvaise{\( \dfrac{1}{24} \)}
              \mauvaise{\( \dfrac{1}{2} \)}
              \mauvaise{\( \dfrac{5}{12} \)}
	      \mauvaise{\( -\dfrac{1}{24} \)}
              \bonne{\( \dfrac{13}{24} \)}
         \end{reponses}
      \end{multicols}
    \end{question}
\end{qcm}

\begin{qcm}{fractions.numeriques.priorites}
    \begin{question}{fractions.numeriques.priorites.2.l}
          Simplifier : \( \dfrac{1}{3}-\dfrac{3}{2}\times\dfrac{1}{4}=\ldots\) 
      \begin{multicols}{3}
         \begin{reponses}
              \mauvaise{\( -\dfrac{7}{24} \)}
              \mauvaise{\( -\dfrac{11}{12} \)}
              \mauvaise{\( -\dfrac{1}{3} \)}
              \bonne{\( -\dfrac{1}{24} \)}
              \mauvaise{\(\dfrac{7}{24}\)}
         \end{reponses}
      \end{multicols}
    \end{question}
\end{qcm}

\begin{qcm}{fractions.numeriques.priorites}
    \begin{question}{fractions.numeriques.priorites.3.l}
          Simplifier : \( \dfrac{4}{3}-\dfrac{1}{2}\times\dfrac{1}{4}=\ldots\) 
      \begin{multicols}{3}
         \begin{reponses}
              \mauvaise{\( \dfrac{5}{24} \)}
              \mauvaise{\( \dfrac{13}{12} \)}
              \mauvaise{\( \dfrac{7}{6} \)}
	      \mauvaise{\( -\dfrac{5}{24} \)}
              \bonne{\( \dfrac{29}{24} \)}
         \end{reponses}
      \end{multicols}
    \end{question}
\end{qcm}




\begin{qcm}{autres.variables.muettes}
    \begin{question}{variables.muettes.simple.1.a}
        Soit \(f\colon x\mapsto\mathrm{e}^{1+x}\). Alors, \(f(3x+1)=\ldots\)
        \vspace{-1.5ex}
         \begin{multicols}{2}
         \begin{reponses}
            \bonne{\(\mathrm{e}^{3x+2}\)}
            \mauvaise{\(\mathrm{e}^{3x+1}\)}
            \mauvaise{\(3\mathrm{e}^{x+1}+1\)}
            \mauvaise{\(3\mathrm{e}^{3x+1}\)}
         \end{reponses}
     \end{multicols}
    \end{question}
\end{qcm}


\begin{qcm}{autres.variables.muettes}
    \begin{question}{variables.muettes.simple.2.a}
        Soit \(f\colon x\mapsto\mathrm{e}^{x-2}\). Alors, \(f(3x-1)=\ldots\)
        \vspace{-1.5ex}
         \begin{multicols}{2}
         \begin{reponses}
            \bonne{\(\mathrm{e}^{3x-3}\)}
            \mauvaise{\(3\mathrm{e}^{x}-1\)}
            \mauvaise{\(3\mathrm{e}^{x-1}-1\)}
            \mauvaise{\(\mathrm{e}^{3x+1}\)}
         \end{reponses}
        \end{multicols}
    \end{question}
\end{qcm}
\begin{qcm}{autres.variations.polynome}
    \begin{question}{variation.1.n}
        La fonction $f\colon x\mapsto -x^2+2x+1$ est $\ldots$
        \vspace{0.1cm}
%         \begin{multicols}{2}
            \begin{reponses}       
                \bonne{décroissante sur $]1,+\infty[$}
                \mauvaise{strictement croissante sur $]1,+\infty[$}
                \mauvaise{croissante sur $[1,+\infty[$}
                \mauvaise{décroissante sur $\mathbb{R}$}
            \end{reponses}
        \vspace{0.4cm}
%         \end{multicols}
    \end{question}
\end{qcm}


\begin{qcm}{autres.variations.polynome}
    \begin{question}{variation.2.n}
        La fonction $f\colon  x\mapsto -x^2+4x-3$ est $\ldots$
        \vspace{0.1cm}
%         \begin{multicols}{2}
            \begin{reponses}       
                \bonne{décroissante sur $]2,+\infty[$}
                \mauvaise{strictement croissante sur $]2,+\infty[$}
                \mauvaise{croissante sur $[2,+\infty[$}
                \mauvaise{décroissante sur $\mathbb{R}$}
            \end{reponses}
        \vspace{0.4cm}
%         \end{multicols}
    \end{question}
\end{qcm}


\begin{qcm}{autres.variations.polynome}
    \begin{question}{variation.3.n}
        La fonction $f\colon  x\mapsto 2x-x^2$ est $\ldots$
        \vspace{0.1cm}
%         \begin{multicols}{2}
            \begin{reponses}       
                \bonne{décroissante sur $]1,+\infty[$}
                \mauvaise{strictement croissante sur $]1,+\infty[$}
                \mauvaise{croissante sur $[1,+\infty[$}
                \mauvaise{strictement décroissante sur $]0,+\infty[$}
            \end{reponses}
        \vspace{0.4cm}
%         \end{multicols}
    \end{question}
\end{qcm}

\begin{qcm}{droites}
    \begin{question}{droites.1.n}
        Dans le plan muni d'un repère $(O;\overrightarrow{i},\overrightarrow{j})$, la droite $\Delta$ d'équation $x+2y+3=0$ coupe l'axe des abscisses au point $A$ d'abscisse:
         \vspace{-1.5ex}
         \begin{multicols}{4}
         \begin{reponses}     
            \bonne{$ -3$}
            \mauvaise{$-3/2$}
            \mauvaise{$3$}
            \mauvaise{$ 0$}
         \end{reponses}
         \end{multicols}
    \end{question}
\end{qcm}


\begin{qcm}{droites}
    \begin{question}{droites.2.n}
        Dans le plan muni d'un repère $(O;\overrightarrow{i},\overrightarrow{j})$, la droite $\Delta$ d'équation $x+2y+3=0$ coupe l'axe des ordonnées au point $A$ d'ordonnée:
         \vspace{-1.5ex}
         \begin{multicols}{4}
         \begin{reponses}         
            \bonne{$-3/2$}
            \mauvaise{$-3$}
            \mauvaise{$3$}
            \mauvaise{$ 0$}
         \end{reponses}
         \end{multicols}
    \end{question}
\end{qcm}


\begin{qcm}{droites}
    \begin{question}{droites.3.n}
        Dans le plan muni d'un repère $(O;\overrightarrow{i},\overrightarrow{j})$, la droite $\Delta$ d'équation $2x+y+3=0$ coupe l'axe des abscisses au point $A$ d'abscisse:
         \vspace{-1.5ex}
         \begin{multicols}{4}
         \begin{reponses}       
            \bonne{$ -3/2$}
            \mauvaise{$-3$}
            \mauvaise{$3$}
            \mauvaise{$ 0$}
         \end{reponses}
         \end{multicols}
    \end{question}
\end{qcm}


\begin{qcm}{droitesbis}
    \begin{question}{droitesbis.1.n}
        Dans un repère $(O;\overrightarrow{i},\overrightarrow{j})$ du plan, on se donne
        les points $A(2;0)$ et $B(0;b)$ où $b$ est un réel non
        nul. Le point d'intersection de la droite $(AB)$ avec la droite
        d'équation $y=3$ a pour coordonnées:
        \vspace{-0.2cm}
        \begin{multicols}{2}
            \begin{reponses}       
                \bonne{$\left(2-\dfrac{6}{b};3\right) $}
                \mauvaise{$\left(2+\dfrac{6}{b};3\right) $}
                \mauvaise{$\left(1+\dfrac{3}{b};3\right) $}
                \mauvaise{$\left(1-\dfrac{3}{b};3\right) $}
                \mauvaise{$\left(-\dfrac{2}{b};3\right) $}
                \alafin
                \mauvaise{Aucune des réponses précédentes}        
            \end{reponses} 
        \end{multicols}
    \end{question}
\end{qcm}

\begin{qcm}{droitesbis}
    \begin{question}{droitesbis.2.n}
        Dans un repère $(O;\overrightarrow{i},\overrightarrow{j})$ du plan, on se donne
        les points $A(a;0)$ et $B(0;2)$ où $a$ est un réel non
        nul. Le point d'intersection de la droite $(AB)$ avec la droite
        d'équation $x=3$ a pour coordonnées:
        \vspace{-0.2cm}
        \begin{multicols}{2}
            \begin{reponses}       
                \bonne{$\left(3;2-\dfrac{6}{a}\right) $}
                \mauvaise{$\left(3;2+\dfrac{6}{a}\right) $}
                \mauvaise{$\left(3;1+\dfrac{3}{a}\right) $}
                \mauvaise{$\left(3;1-\dfrac{3}{a}\right) $}
                \mauvaise{$\left(3;-\dfrac{2}{a}\right) $}
                \alafin
                \mauvaise{Aucune des réponses précédentes}        
            \end{reponses} 
        \end{multicols}
    \end{question}
\end{qcm}


\begin{qcm}{trigonometrie.valeurs}
    \begin{question}{trigo.valeurs.1.k}
         \({\cos\left(\frac{3\pi}{4}\right)}=\ldots\)
         \vspace{-1.5ex}
         \begin{multicols}{3}
         \begin{reponses}
          \lastchoices
              \mauvaise{\(-\dfrac{\sqrt{3}}{2}\)}
              \bonne{\(-\dfrac{\sqrt{2}}{2}\)}
              \mauvaise{\(\dfrac{1}{2} \textcolor{white}{\dfrac{\sqrt{3}}{1}}\)}
              \mauvaise{\(\dfrac{\sqrt{2}}{2}\)}
              \mauvaise{\(\dfrac{\sqrt{3}}{2}\)}
         \end{reponses}
        \end{multicols}
    \end{question}
\end{qcm}




\begin{qcm}{trigonometrie.valeurs}
    \begin{question}{trigo.valeurs.2.k}
         \({\cos\left(-\frac{3\pi}{4}\right)}=\ldots\)
         \vspace{-1.5ex}
         \begin{multicols}{3}
         \begin{reponses}
          \lastchoices
              \mauvaise{\(-\dfrac{\sqrt{3}}{2}\)}
              \bonne{\(-\dfrac{\sqrt{2}}{2}\)}
              \mauvaise{\(\dfrac{1}{2} \textcolor{white}{\dfrac{\sqrt{3}}{1}}\)}
              \mauvaise{\(\dfrac{\sqrt{2}}{2}\)}
              \mauvaise{\(\dfrac{\sqrt{3}}{2}\)}
         \end{reponses}
        \end{multicols}
    \end{question}
\end{qcm}

\begin{qcm}{trigonometrie.valeurs}
    \begin{question}{trigo.valeurs.3.k}
         \({\cos\left(\frac{3\pi}{2}\right)}=\ldots\)
         \vspace{-1.5ex}
         \begin{multicols}{3}
            \begin{reponses}
                \lastchoices
                \mauvaise{\(-1 \textcolor{white}{\dfrac{\sqrt{3}}{1}}\)}
                \mauvaise{\(-\dfrac{\sqrt{2}}{2}\)}
                \bonne{\(0 \textcolor{white}{\dfrac{\sqrt{3}}{1}}\)}
                \mauvaise{\(\dfrac{1}{2} \textcolor{white}{\dfrac{\sqrt{3}}{1}}\)}
                \mauvaise{\(\dfrac{\sqrt{2}}{2}\)}
            \end{reponses}
        \end{multicols}
    \end{question}
\end{qcm}

\begin{qcm}{trigonometrie.valeurs}
    \begin{question}{trigo.valeurs.4.k}
         \({\cos\left(-\frac{3\pi}{2}\right)}=\ldots\)
         \vspace{-1.5ex}
         \begin{multicols}{3}
            \begin{reponses}
                \lastchoices
                \mauvaise{\(-1 \textcolor{white}{\dfrac{\sqrt{3}}{1}}\)}
                \mauvaise{\(-\dfrac{\sqrt{2}}{2}\)}
                \bonne{\(0 \textcolor{white}{\dfrac{\sqrt{3}}{1}}\)}
                \mauvaise{\(\dfrac{1}{2} \textcolor{white}{\dfrac{\sqrt{3}}{1}}\)}
                \mauvaise{\(\dfrac{\sqrt{2}}{2}\)}
            \end{reponses}
        \end{multicols}
    \end{question}
\end{qcm}

\begin{qcm}{trigonometrie.valeurs}
    \begin{question}{trigo.valeurs.5.k}
         \({\cos\left(\frac{2\pi}{3}\right)}=\ldots\)
         \vspace{-1.5ex}
         \begin{multicols}{3}
            \begin{reponses}
                \lastchoices
                \mauvaise{\(-\dfrac{\sqrt{3}}{2}\)}
                \bonne{\(-\dfrac{1}{2} \textcolor{white}{\dfrac{\sqrt{3}}{1}}\)}
                \mauvaise{\(0 \textcolor{white}{\dfrac{\sqrt{3}}{2}}\)}
                \mauvaise{\(\dfrac{1}{2} \textcolor{white}{\dfrac{\sqrt{3}}{1}}\)}
                \mauvaise{\(\dfrac{\sqrt{3}}{2}\)}
            \end{reponses}
        \end{multicols}
    \end{question}
\end{qcm}

\begin{qcm}{trigonometrie.valeurs}
    \begin{question}{trigo.valeurs.6.k}
         \({\cos\left(-\frac{2\pi}{3}\right)}=\ldots\)
         \vspace{-1.5ex}
         \begin{multicols}{3}
            \begin{reponses}
                \lastchoices
                \mauvaise{\(-\dfrac{\sqrt{3}}{2}\)}
                \bonne{\(-\dfrac{1}{2} \textcolor{white}{\dfrac{\sqrt{3}}{1}}\)}
                \mauvaise{\(0 \textcolor{white}{\dfrac{\sqrt{3}}{2}}\)}
                \mauvaise{\(\dfrac{1}{2} \textcolor{white}{\dfrac{\sqrt{3}}{1}}\)}
                \mauvaise{\(\dfrac{\sqrt{3}}{2}\)}
            \end{reponses}
        \end{multicols}
    \end{question}
\end{qcm}

\begin{qcm}{trigonometrie.valeurs}
    \begin{question}{trigo.valeurs.7.k}
         \({\cos\left(\frac{5\pi}{6}\right)}=\ldots\)
         \vspace{-1.5ex}
         \begin{multicols}{3}
            \begin{reponses}
                \lastchoices
                \bonne{\(-\dfrac{\sqrt{3}}{2}\)}
                \mauvaise{\(-\dfrac{1}{2} \textcolor{white}{\dfrac{\sqrt{3}}{1}}\)}
                \mauvaise{\(\dfrac{1}{2} \textcolor{white}{\dfrac{\sqrt{3}}{1}}\)}
                \mauvaise{\(\dfrac{\sqrt{2}}{2}\)}
                \mauvaise{\(\dfrac{\sqrt{3}}{2}\)}
%               \mauvaise{\(\dfrac{-\sqrt{2}}{2}\)}
            \end{reponses}
        \end{multicols}
    \end{question}
\end{qcm}

\begin{qcm}{trigonometrie.valeurs}
    \begin{question}{trigo.valeurs.8.k}
         \({\cos\left(-\frac{5\pi}{6}\right)}=\ldots\)
         \vspace{-1.5ex}
         \begin{multicols}{3}
            \begin{reponses}
                \lastchoices
                \bonne{\(-\dfrac{\sqrt{3}}{2}\)}
                \mauvaise{\(-\dfrac{\sqrt{2}}{2}\)}
                \mauvaise{\(-\dfrac{1}{2} \textcolor{white}{\dfrac{\sqrt{3}}{1}}\)}
                \mauvaise{\(\dfrac{1}{2} \textcolor{white}{\dfrac{\sqrt{3}}{1}}\)}
                \mauvaise{\(\dfrac{\sqrt{3}}{2}\)}
            \end{reponses}
        \end{multicols}
    \end{question}
\end{qcm}

\begin{qcm}{trigonometrie.valeurs}
    \begin{question}{trigo.valeurs.9.k}
         \({\sin\left(\frac{3\pi}{4}\right)}=\ldots\)
         \vspace{-1.5ex}
         \begin{multicols}{3}
            \begin{reponses}
                \lastchoices
                \mauvaise{\(0 \textcolor{white}{\dfrac{\sqrt{3}}{1}}\)}
                \mauvaise{\(\dfrac{1}{2} \textcolor{white}{\dfrac{\sqrt{3}}{1}}\)}
                \mauvaise{\(1 \textcolor{white}{\dfrac{\sqrt{3}}{1}}\)}
                \bonne{\(\dfrac{\sqrt{2}}{2}\)}
                \mauvaise{\(\dfrac{\sqrt{3}}{2}\)}
            \end{reponses}
        \end{multicols}
    \end{question}
\end{qcm}

\begin{qcm}{trigonometrie.valeurs}
    \begin{question}{trigo.valeurs.10.k}
         \({\sin\left(-\frac{3\pi}{4}\right)}=\ldots\)
         \vspace{-1.5ex}
         \begin{multicols}{3}
            \begin{reponses}
                \lastchoices
                \mauvaise{\(-\dfrac{\sqrt{3}}{2}\)}
                \bonne{\(-\dfrac{\sqrt{2}}{2}\)}
                \mauvaise{\(-\dfrac{1}{2}\textcolor{white}{\dfrac{\sqrt{3}}{1}}\)}
                \mauvaise{\(\dfrac{\sqrt{2}}{2}\)}
                \mauvaise{\(\dfrac{\sqrt{3}}{2}\)}              
            \end{reponses}
        \end{multicols}
    \end{question}
\end{qcm}

\begin{qcm}{trigonometrie.valeurs}
    \begin{question}{trigo.valeurs.11.k}
         \({\sin\left(\frac{3\pi}{2}\right)}=\ldots\)
         \vspace{-1.5ex}
         \begin{multicols}{3}
            \begin{reponses}
                \lastchoices
                \bonne{\(-1 \textcolor{white}{\dfrac{\sqrt{3}}{1}}\)}
                \mauvaise{\(-\dfrac{1}{2} \textcolor{white}{\dfrac{\sqrt{3}}{1}}\)}
                \mauvaise{\(0 \textcolor{white}{\dfrac{\sqrt{3}}{1}}\)}
                \mauvaise{\(1 \textcolor{white}{\dfrac{\sqrt{3}}{1}}\)}
                \mauvaise{\(\dfrac{\sqrt{3}}{2}\)}
            \end{reponses}
        \end{multicols}
    \end{question}
\end{qcm}

\begin{qcm}{trigonometrie.valeurs}
    \begin{question}{trigo.valeurs.12.k}
         \({\sin\left(-\frac{3\pi}{2}\right)}=\ldots\)
         \vspace{-1.5ex}
         \begin{multicols}{3}
            \begin{reponses}
                \lastchoices
                \mauvaise{\(-1 \textcolor{white}{\dfrac{\sqrt{3}}{1}}\)}
                \mauvaise{\(-\dfrac{\sqrt{3}}{2}\)}
                \mauvaise{\(-\dfrac{1}{2} \textcolor{white}{\dfrac{\sqrt{3}}{1}}\)}
                \mauvaise{\(0 \textcolor{white}{\dfrac{\sqrt{3}}{1}}\)}              
                \bonne{\(1 \textcolor{white}{\dfrac{\sqrt{3}}{1}}\)}
            \end{reponses}
        \end{multicols}
    \end{question}
\end{qcm}

\begin{qcm}{trigonometrie.valeurs}
    \begin{question}{trigo.valeurs.13.k}
         \({\sin\left(\frac{2\pi}{3}\right)}=\ldots\)
         \vspace{-1.5ex}
         \begin{multicols}{3}
            \begin{reponses}
                \lastchoices
                \mauvaise{\(-\dfrac{1}{2} \textcolor{white}{\dfrac{\sqrt{3}}{1}}\)}
                \mauvaise{\(0 \textcolor{white}{\dfrac{\sqrt{3}}{1}}\)}
                \mauvaise{\(\dfrac{1}{2} \textcolor{white}{\dfrac{\sqrt{3}}{1}}\)}
                \mauvaise{\(\dfrac{\sqrt{2}}{2}\)}
                \bonne{\(\dfrac{\sqrt{3}}{2}\)}
            \end{reponses}
        \end{multicols}
    \end{question}
\end{qcm}

\begin{qcm}{trigonometrie.valeurs}
    \begin{question}{trigo.valeurs.14.k}
         \({\sin\left(-\frac{2\pi}{3}\right)}=\ldots\)
         \vspace{-1.5ex}
         \begin{multicols}{3}
            \begin{reponses}
                \lastchoices
                \bonne{\(-\dfrac{\sqrt{3}}{2}\)}
                \mauvaise{\(-\dfrac{1}{2} \textcolor{white}{\dfrac{\sqrt{3}}{1}}\)}
                \mauvaise{\(\dfrac{1}{2} \textcolor{white}{\dfrac{\sqrt{3}}{1}}\)}
                \mauvaise{\(\dfrac{\sqrt{2}}{2}\)}
                \mauvaise{\(\dfrac{\sqrt{3}}{2}\)}
            \end{reponses}
        \end{multicols}
    \end{question}
\end{qcm}

\begin{qcm}{trigonometrie.valeurs}
    \begin{question}{trigo.valeurs.15.k}
         \({\sin\left(\frac{5\pi}{6}\right)}=\ldots\)
         \vspace{-1.5ex}
         \begin{multicols}{3}
            \begin{reponses}
                \lastchoices
                \mauvaise{\(-\dfrac{\sqrt{3}}{2}\)}
                \mauvaise{\(0 \textcolor{white}{\dfrac{\sqrt{3}}{1}}\)}
                \bonne{\(\dfrac{1}{2} \textcolor{white}{\dfrac{\sqrt{3}}{1}}\)}
                \mauvaise{\(\dfrac{\sqrt{2}}{2}\)}
                \mauvaise{\(\dfrac{\sqrt{3}}{2}\)}
            \end{reponses}
        \end{multicols}
    \end{question}
\end{qcm}

\begin{qcm}{trigonometrie.valeurs}
    \begin{question}{trigo.valeurs.16.k}
         \({\sin\left(-\frac{5\pi}{6}\right)}=\ldots\)
         \vspace{-1.5ex}
         \begin{multicols}{3}
            \begin{reponses}
                \lastchoices
                \mauvaise{\(-\dfrac{\sqrt{3}}{2}\)}
                \bonne{\(-\dfrac{1}{2} \textcolor{white}{\dfrac{\sqrt{3}}{1}}\)}
                \mauvaise{\(\dfrac{1}{2} \textcolor{white}{\dfrac{\sqrt{3}}{1}}\)}
                \mauvaise{\(\dfrac{\sqrt{2}}{2}\)}
                \mauvaise{\(\dfrac{\sqrt{3}}{2}\)}
            \end{reponses}
        \end{multicols}
    \end{question}
\end{qcm}

\begin{qcm}{trigonometrie.symetries}
    \begin{question}{trigo.symetries.1.k}
         Soit \(x\) un nombre réel. \(\sin\left(\pi-x\right)=\ldots\)
         \vspace{-1.5ex}
         \begin{multicols}{2}
         \begin{reponses}
              \mauvaise{\(\cos(x)\)}
              \mauvaise{\(-\cos(x)\)}
              \bonne{\(\sin(x)\)}
              \mauvaise{\(-\sin(x)\)}
         \end{reponses}
        \end{multicols}
    \end{question}
\end{qcm}

\begin{qcm}{trigonometrie.symetries}
    \begin{question}{trigo.symetries.2.k}
         Soit \(x\) un nombre réel. \(\cos\left(\pi-x\right)=\ldots\)
         \vspace{-1.5ex}
         \begin{multicols}{2}
         \begin{reponses}
              \mauvaise{\(\cos(x)\)}
              \bonne{\(-\cos(x)\)}
              \mauvaise{\(\sin(x)\)}
              \mauvaise{\(-\sin(x)\)}
         \end{reponses}
        \end{multicols}
    \end{question}
\end{qcm}

\begin{qcm}{trigonometrie.symetries}
    \begin{question}{trigo.symetries.3.k}
         Soit \(x\) un nombre réel. \(\sin\left(x+\frac{\pi}{2}\right)=\ldots\)
         \vspace{-1.5ex}
         \begin{multicols}{2}
         \begin{reponses}
              \mauvaise{\(\cos(x)\)}
              \mauvaise{\(-\cos(x)\)}
              \mauvaise{\(\sin(x)\)}
              \bonne{\(-\sin(x)\)}
         \end{reponses}
        \end{multicols}
    \end{question}
\end{qcm}

\begin{qcm}{trigonometrie.symetries}
    \begin{question}{trigo.symetries.4.k}
         Soit \(x\) un nombre réel. \(\cos\left(x+\frac{\pi}{2}\right)=\ldots\)
         \vspace{-1.5ex}
         \begin{multicols}{2}
         \begin{reponses}
              \mauvaise{\(\cos(x)\)}
              \mauvaise{\(-\cos(x)\)}
              \mauvaise{\(\sin(x)\)}
              \bonne{\(-\sin(x)\)}
         \end{reponses}
        \end{multicols}
    \end{question}
\end{qcm}

\begin{qcm}{trigonometrie.lienSinusCosinus}
    \begin{question}{trigo.lienSinusCosinus.1.k}
         On considère un nombre réel \(x\in\left[2\pi,\frac{5\pi}{2}\right]\) tel que \(\cos(x)=\frac{4}{5}\). Que vaut \(\sin(x)\) ?
         \vspace{-1.5ex}
         \begin{multicols}{4}
         \begin{reponses}
              \mauvaise{\(\dfrac{\sqrt{17}}{5}\)}
              \mauvaise{\(-\dfrac{3}{5}\)}
              \bonne{\(\dfrac{3}{5}\)}
              \mauvaise{\(\dfrac{9}{25}\)}
         \end{reponses}
        \end{multicols}
    \end{question}
\end{qcm}

\begin{qcm}{trigonometrie.lienSinusCosinus}
    \begin{question}{trigo.lienSinusCosinus.2.k}
         On considère un nombre réel \(x\in\left[\frac{3\pi}{2},2\pi\right]\) tel que \(\cos(x)=\frac{4}{5}\). Que vaut \(\sin(x)\) ?
         \vspace{-1.5ex}
         \begin{multicols}{4}
         \begin{reponses}
              \mauvaise{\(\dfrac{\sqrt{17}}{5}\)}
              \bonne{\(-\dfrac{3}{5}\)}
              \mauvaise{\(\dfrac{3}{5}\)}
              \mauvaise{\(\dfrac{9}{25}\)}
         \end{reponses}
        \end{multicols}
    \end{question}
\end{qcm}

\begin{qcm}{trigonometrie.lienSinusCosinus}
    \begin{question}{trigo.lienSinusCosinus.3.k}
         On considère un nombre réel \(x\in\left[3\pi,\frac{7\pi}{2}\right]\) tel que \(\cos(x)=-\frac{4}{5}\). Que vaut \(\sin(x)\) ?
         \vspace{-1.5ex}
         \begin{multicols}{4}
         \begin{reponses}
              \mauvaise{\(\dfrac{\sqrt{17}}{5}\)}
              \bonne{\(-\dfrac{3}{5}\)}
              \mauvaise{\(\dfrac{3}{5}\)}
              \mauvaise{\(\dfrac{9}{25}\)}
         \end{reponses}
        \end{multicols}
    \end{question}
\end{qcm}

\begin{qcm}{trigonometrie.lienSinusCosinus}
    \begin{question}{trigo.lienSinusCosinus.4.k}
         On considère un nombre réel \(x\in\left[\frac{5\pi}{2},3\pi\right]\) tel que \(\cos(x)=-\frac{4}{5}\). Que vaut \(\sin(x)\) ?
         \vspace{-1.5ex}
         \begin{multicols}{4}
         \begin{reponses}
              \mauvaise{\(\dfrac{\sqrt{17}}{5}\)}
              \mauvaise{\(-\dfrac{3}{5}\)}
              \bonne{\(\dfrac{3}{5}\)}
              \mauvaise{\(\dfrac{9}{25}\)}
         \end{reponses}
        \end{multicols}
    \end{question}
\end{qcm}

\begin{qcm}{trigonometrie.encadrements}
    \begin{question}{trigonometrie.encadrements.1.k}
         Soit \(0<x<\frac{\pi}{8}\). Encadrer \(\frac{1}{\cos^2(2x)}\) par deux entiers consécutifs : 
         \( n<\frac{1}{\cos^2(2x)}<n+1\) où \(n=\ldots\)
         \vspace{-0.1cm}
         \begin{reponseshoriz}
            \mauvaise{\(-1\)}
            \mauvaise{\(0\)}
            \bonne{\(1\)}
            \mauvaise{\(2\)}
         \end{reponseshoriz}
    \end{question}
\end{qcm}

\begin{qcm}{trigonometrie.encadrements}
    \begin{question}{trigonometrie.encadrements.2.k}
         Soit \(0<x<\frac{\pi}{4}\). Encadrer \(\frac{1}{\cos^2(x)}\) par deux entiers consécutifs : 
         \( n<\frac{1}{\cos^2(x)}<n+1\) où \(n=\ldots\)
         \vspace{-0.1cm}
         \begin{reponseshoriz}
            \mauvaise{\(-1\)}
            \mauvaise{\(0\)}
            \bonne{\(1\)}
            \mauvaise{\(2\)}
         \end{reponseshoriz}
    \end{question}
\end{qcm}

\begin{qcm}{trigonometrie.encadrements}
    \begin{question}{trigonometrie.encadrements.3.k}
         Soit \(\frac{\pi}{4}<x<\frac{\pi}{2}\). Encadrer \(\frac{1}{\sin^2(x)}\) par deux entiers consécutifs : 
         \( n<\frac{1}{\sin^2(x)}<n+1\) où \(n=\ldots\)
         \vspace{-0.1cm}
         \begin{reponseshoriz}
            \mauvaise{\(-1\)}
            \mauvaise{\(0\)}
            \bonne{\(1\)}
            \mauvaise{\(2\)}
         \end{reponseshoriz}
    \end{question}
\end{qcm}

\begin{qcm}{trigonometrie.encadrements}
    \begin{question}{trigonometrie.encadrements.3.k}
         Soit \(\frac{\pi}{8}<x<\frac{\pi}{4}\). Encadrer \(\frac{1}{\sin^2(2x)}\) par deux entiers consécutifs : 
         \( n<\frac{1}{\sin^2(2x)}<n+1\) où \(n=\ldots\)
         \vspace{-0.3cm}
         \begin{reponseshoriz}
            \mauvaise{\(-1\)}
            \mauvaise{\(0\)}
            \bonne{\(1\)}
            \mauvaise{\(2\)}
         \end{reponseshoriz}
    \end{question}
\end{qcm}


\begin{qcm}{puissances.regles}
  \begin{question}{puissances.regles.1.k}
    Soient \(x\), \(p\) et \(q\) trois nombres réels avec \(x>0\). Alors \({\left(x^p\right)}^q=x^r\) où \(r=\ldots\)
    \vspace{-1.5ex}
    \begin{multicols}{3}
      \begin{reponses}
        \lastchoices
        \mauvaise{\(p+q\)}
        \mauvaise{\(p-q\)}
        \bonne{\(p\times q\)}
        \mauvaise{\(p^q\)}
        \mauvaise{\(\dfrac{p}{q}\)}

        \phantom{foo}
      \end{reponses}
    \end{multicols}
  \end{question}
\end{qcm}

\begin{qcm}{puissances.regles}
  \begin{question}{puissances.regles.2.k}
    Soient \(x\), \(p\) et \(q\) trois nombres réels avec \(x>0\). \(x^p \times x^q=x^r\) où \(r=\ldots\)
    \vspace{-1.5ex}
    \begin{multicols}{3}
      \begin{reponses}
        \lastchoices
        \bonne{\(p+q\)}
        \mauvaise{\(p-q\)}
        \mauvaise{\(p\times q\)}
        \mauvaise{\(p^q\)}
        \mauvaise{\(\dfrac{p}{q}\)}

        \phantom{foo}
      \end{reponses}
    \end{multicols}
  \end{question}
\end{qcm}

\begin{qcm}{puissances.regles}
  \begin{question}{puissances.regles.3.k}
    Soient \(x\), \(p\) et \(q\) trois nombres réels avec \(x>0\). Alors \(\dfrac{x^p}{x^q}=x^r\) où \(r=\ldots\)
    \vspace{-1.5ex}
    \begin{multicols}{3}
      \begin{reponses}
        \lastchoices
        \mauvaise{\(p+q\)}
        \mauvaise{\(p-q\)}
        \bonne{\(p\times q\)}
        \mauvaise{\(p^q\)}
        \mauvaise{\(\dfrac{p}{q}\)}

        \phantom{foo}
      \end{reponses}
    \end{multicols}
  \end{question}
\end{qcm}

\begin{qcm}{puissances.regles}
  \begin{question}{puissances.regles.4.k}
    {Soient \(x\), \(y\) et \(p\) des nombres réels avec \(x>0\) et \(y>0\). \(\left(\dfrac{x}{y}\right)^p=\ldots\)
      %          \vspace{-1.5ex}
      \begin{multicols}{2}
        \begin{reponses}
          \lastchoices
          \mauvaise{\(x^p \times y^{\frac{1}{p}} \)}
          \mauvaise{\(\dfrac{x^p}{y^{\frac{1}{p}}}\)}
          \bonne{\(x^p \times y^{-p} \)}
          \mauvaise{\(\dfrac{x^p}{y^{-p}} \textcolor{white}{\dfrac{x^p}{y^{\frac{1}{p}}}}\)}
        \end{reponses}
      \end{multicols}}
  \end{question}
\end{qcm}


\begin{qcm}{puissances.simplifications}
    \begin{questionmult}{puissances.simplifications.1.k}
	Si \(x>0\), alors \(\dfrac{x^2\times x^{-6} \times x^9}{x^{-3}\times {\sqrt{x^{10}}}}=x^p\) où \(p=\ldots\)
	
	\noindent\textit{Cocher ci-dessous le signe de \(p\) et sa valeur.}
	\vspace{-1.2ex}
	\footnotesize{
	 \AMCnumericChoices{3}{digits=1,decimals=0,sign=true,borderwidth=0pt,backgroundcol=white,approx=0,hspace=0.4em,vspace=0.5ex}}
    \end{questionmult}
\end{qcm}

\begin{qcm}{puissances.simplifications}
    \begin{questionmult}{puissances.simplifications.2.k}
	Si \(x \neq 0\), alors \(\dfrac{x^{-3}\times x^5}{{\sqrt{x^4}}\times x^{-6} \times x^9}=x^p\) où \(p=\ldots\)
    
    \noindent\textit{Cocher ci-dessous le signe de \(p\) et sa valeur.}
    \vspace{-1.2ex}
    \footnotesize{
    \AMCnumericChoices{-3}{digits=1,decimals=0,sign=true,borderwidth=0pt,backgroundcol=white,approx=0,hspace=0.4em,vspace=0.5ex}}
    \end{questionmult}
\end{qcm}

\begin{qcm}{puissances.simplifications}
    \begin{questionmult}{puissances.simplifications.3.k}
	Si \(x \neq 0\), alors \(\dfrac{x^7\times x^{-2} \times {\sqrt{x^6}}}{x^{-4}\times x^8 \times x^5}=x^p\) où \(p=\ldots\)
    
    \noindent\textit{Cocher ci-dessous le signe de \(p\) et sa valeur.}
    \vspace{-1.2ex}
    \footnotesize{
    	  \AMCnumericChoices{-1}{digits=1,decimals=0,sign=true,borderwidth=0pt,backgroundcol=white,approx=0,hspace=0.4em,vspace=0.5ex}}
    \end{questionmult}
\end{qcm}

\begin{qcm}{puissances.simplifications}
    \begin{questionmult}{puissances.simplifications.4.k}
	Si \(x \neq 0\), alors \(\dfrac{x^{-4}\times x^8 \times {\sqrt{x^{10}}}}{x^7\times x^{-2} \times x^3}=x^p\) où \(p=\ldots\)
    
    \noindent\textit{Cocher ci-dessous le signe de \(p\) et sa valeur.}
    \vspace{-1.2ex}
    \footnotesize{
    	  \AMCnumericChoices{1}{digits=1,decimals=0,sign=true,borderwidth=0pt,backgroundcol=white,approx=0,hspace=0.4em,vspace=0.5ex}}
    \end{questionmult}
\end{qcm}

\begin{qcm}{puissances.simplifications}
    \begin{questionmult}{puissances.simplifications.5.k}
	Si \(x \neq 0\), alors \(\dfrac{x^{-8}\times {\sqrt{x^6}}}{x^{-4}\times x^{10} \times x^{-2}}=x^p\) où \(p=\ldots\)
	
    \noindent\textit{Cocher ci-dessous le signe de \(p\) et sa valeur.}
    \vspace{-1.2ex}
    \footnotesize{
    	  \AMCnumericChoices{-9}{digits=1,decimals=0,sign=true,borderwidth=0pt,backgroundcol=white,approx=0,hspace=0.4em,vspace=0.5ex}}
    \end{questionmult}
\end{qcm}

\begin{qcm}{puissances.simplifications}
    \begin{questionmult}{puissances.simplifications.6.k}
	Si \(x \neq 0\), alors \(\dfrac{x^{-4}\times x^{10} \times x^{-2}}{x^{-8}\times {\sqrt{x^6}}}=x^p\) où \(p=\ldots\)
    
    \noindent\textit{Cocher ci-dessous le signe de \(p\) et sa valeur.}
    \vspace{-1.2ex}
    \footnotesize{
    	  \AMCnumericChoices{9}{digits=1,decimals=0,sign=true,borderwidth=0pt,backgroundcol=white,approx=0,hspace=0.4em,vspace=0.5ex}}
    \end{questionmult}
\end{qcm}


  
  


\begin{qcm}{puissances.simplifications}
    \begin{questionmult}{puissances.simplifications.1.k}
	Si \(x>0\), alors \(\dfrac{x^2\times x^{-6} \times x^9}{x^{-3}\times {\sqrt{x^{10}}}}=x^p\) où \(p=\ldots\)
	
	\noindent\textit{Cocher ci-dessous le signe de \(p\) et sa valeur.}
	\vspace{-1.2ex}
	\footnotesize{
	 \AMCnumericChoices{3}{digits=1,decimals=0,sign=true,borderwidth=0pt,backgroundcol=white,approx=0,hspace=0.4em,vspace=0.5ex}}
    \end{questionmult}
\end{qcm}

\begin{qcm}{puissances.simplifications}
    \begin{questionmult}{puissances.simplifications.2.k}
	Si \(x \neq 0\), alors \(\dfrac{x^{-3}\times x^5}{{\sqrt{x^4}}\times x^{-6} \times x^9}=x^p\) où \(p=\ldots\)
    
    \noindent\textit{Cocher ci-dessous le signe de \(p\) et sa valeur.}
    \vspace{-1.2ex}
    \footnotesize{
    \AMCnumericChoices{-3}{digits=1,decimals=0,sign=true,borderwidth=0pt,backgroundcol=white,approx=0,hspace=0.4em,vspace=0.5ex}}
    \end{questionmult}
\end{qcm}

\begin{qcm}{puissances.simplifications}
    \begin{questionmult}{puissances.simplifications.3.k}
	Si \(x \neq 0\), alors \(\dfrac{x^7\times x^{-2} \times {\sqrt{x^6}}}{x^{-4}\times x^8 \times x^5}=x^p\) où \(p=\ldots\)
    
    \noindent\textit{Cocher ci-dessous le signe de \(p\) et sa valeur.}
    \vspace{-1.2ex}
    \footnotesize{
    	  \AMCnumericChoices{-1}{digits=1,decimals=0,sign=true,borderwidth=0pt,backgroundcol=white,approx=0,hspace=0.4em,vspace=0.5ex}}
    \end{questionmult}
\end{qcm}

\begin{qcm}{puissances.simplifications}
    \begin{questionmult}{puissances.simplifications.4.k}
	Si \(x \neq 0\), alors \(\dfrac{x^{-4}\times x^8 \times {\sqrt{x^{10}}}}{x^7\times x^{-2} \times x^3}=x^p\) où \(p=\ldots\)
    
    \noindent\textit{Cocher ci-dessous le signe de \(p\) et sa valeur.}
    \vspace{-1.2ex}
    \footnotesize{
    	  \AMCnumericChoices{1}{digits=1,decimals=0,sign=true,borderwidth=0pt,backgroundcol=white,approx=0,hspace=0.4em,vspace=0.5ex}}
    \end{questionmult}
\end{qcm}

\begin{qcm}{puissances.simplifications}
    \begin{questionmult}{puissances.simplifications.5.k}
	Si \(x \neq 0\), alors \(\dfrac{x^{-8}\times {\sqrt{x^6}}}{x^{-4}\times x^{10} \times x^{-2}}=x^p\) où \(p=\ldots\)
	
    \noindent\textit{Cocher ci-dessous le signe de \(p\) et sa valeur.}
    \vspace{-1.2ex}
    \footnotesize{
    	  \AMCnumericChoices{-9}{digits=1,decimals=0,sign=true,borderwidth=0pt,backgroundcol=white,approx=0,hspace=0.4em,vspace=0.5ex}}
    \end{questionmult}
\end{qcm}

\begin{qcm}{puissances.simplifications}
    \begin{questionmult}{puissances.simplifications.6.k}
	Si \(x \neq 0\), alors \(\dfrac{x^{-4}\times x^{10} \times x^{-2}}{x^{-8}\times {\sqrt{x^6}}}=x^p\) où \(p=\ldots\)
    
    \noindent\textit{Cocher ci-dessous le signe de \(p\) et sa valeur.}
    \vspace{-1.2ex}
    \footnotesize{
    	  \AMCnumericChoices{9}{digits=1,decimals=0,sign=true,borderwidth=0pt,backgroundcol=white,approx=0,hspace=0.4em,vspace=0.5ex}}
    \end{questionmult}
\end{qcm}


  
  



\begin{qcm}{derivees.usuelles.composee}
    \begin{question}{derivees.usuelles.composee.1}
         Posons pour tout nombre réel \(x\), \( f(x)=\mathrm{e}^{-x} \). Alors \( f'(x)=\ldots\)
         \begin{multicols}{3}     
	    \begin{reponses}
            \lastchoices
	          \mauvaise{\(-\ln(x)\)}
              \mauvaise{\( -\mathrm{e}^{x} \)}
              \mauvaise{\( \mathrm{e}^{-x} \)}
              \bonne{\( - \mathrm{e}^{-x} \)}
              \mauvaise{\( \dfrac{1}{x} \)}
  		 
             \phantom{foo}
          \end{reponses}
         \end{multicols}
    \end{question}
\end{qcm}

\begin{qcm}{derivees.usuelles.composee}
    \begin{question}{derivees.usuelles.composee.2}
         Posons pour tout nombre réel \(x\), \( f(x)=\mathrm{e}^{2x} \). Alors \( f'(x)=\ldots\)
         \begin{multicols}{3}     
	    \begin{reponses}
              \lastchoices
	          \mauvaise{\(\ln(2x)\)}
              \mauvaise{\( \dfrac{1}{2x} \)}
              \mauvaise{\( 2 \mathrm{e}^{x} \)}
              \mauvaise{\( \mathrm{e}^{2x} \)}
              \bonne{\( 2 \mathrm{e}^{2x} \)}
  		 
  		 \phantom{foo}
	    \end{reponses}
         \end{multicols}
    \end{question}
\end{qcm}

\begin{qcm}{derivees.usuelles.composee}
    \begin{question}{derivees.usuelles.composee.3}
         Posons pour tout nombre réel \(x\) strictment négatif, \( f(x)= \ln(-x) \). Alors \( f'(x)=\ldots\)
         \begin{multicols}{3}     
	    \begin{reponses}
              \mauvaise{\( - \dfrac{1}{x} \)}
              \mauvaise{\( \dfrac{1}{2x}  \)}
%               \bonne{\( \dfrac{-1}{-x}  \)}
		 \bonne{\( \dfrac{1}{x}  \)}
	      \mauvaise{\(\mathrm{e}^{-x} \textcolor{white}{\dfrac{e^a}{b}}\)}
	      \mauvaise{\(-\mathrm{e}^{-x}\textcolor{white}{\dfrac{e^a}{b}}\)}
  		 \lastchoices
  		 
  		 \phantom{foo}
	    \end{reponses}
         \end{multicols}
    \end{question}
\end{qcm}

\begin{qcm}{derivees.usuelles.composee}
    \begin{question}{derivees.usuelles.composee.4}
        Posons pour tout nombre réel \(x\) strictement positif, \( f(x)=\ln(2x) \). Alors \( f'(x)=\ldots\)
         \begin{multicols}{3}     
	    \begin{reponses}
              \mauvaise{\( \dfrac{2}{x} \)}
              \mauvaise{\( \dfrac{1}{2x}  \)}
              \bonne{\( \dfrac{1}{x}  \)}
              \mauvaise{\(\mathrm{e}^{2x} \textcolor{white}{\dfrac{e^a}{b}}\)}
              \mauvaise{\(2\mathrm{e}^{x} \textcolor{white}{\dfrac{e^a}{b}}\)}
  		 \lastchoices
  		 
  		 \phantom{foo}
	    \end{reponses}
         \end{multicols}
    \end{question}
\end{qcm}

\begin{qcm}{derivees.usuelles.composee}
    \begin{question}{derivees.usuelles.composee.5}
         Posons pour tout nombre réel \(x\), \( f(x)=\cos(2x) \). Alors \( f'(x)=\ldots\)
         \begin{multicols}{2}     
	    \begin{reponses}
              \mauvaise{\( 2 \cos(2x) \)}
              \mauvaise{\( \sin(2x)  \)}
	      \mauvaise{\( - \sin(2x)  \)}
	      \mauvaise{\( 2 \sin(2x)  \)}
              \bonne{\( -2 \sin(2x)  \)}
  		 \lastchoices
  		 
  		 \phantom{foo}
          \end{reponses}
         \end{multicols}
    \end{question}
\end{qcm}

\begin{qcm}{derivees.usuelles.composee}
    \begin{question}{derivees.usuelles.composee.6}
          Posons pour tout nombre réel \(x\), \( f(x)=\sin(2x) \). Alors \( f'(x)=\ldots\)
         \begin{multicols}{2}     
	    \begin{reponses}
              \mauvaise{\( 2 \sin(2x) \)}
              \mauvaise{\( \cos(2x)  \)}
  		 \mauvaise{\( - \cos(2x)  \)}
           	 \mauvaise{\( - 2 \cos(2x)  \)}
              \bonne{\( 2 \cos(2x)  \)}
  		 \lastchoices
  		 
  		 \phantom{foo}
	    \end{reponses}
         \end{multicols}
    \end{question}
\end{qcm} 

\begin{qcm}{derivees.usuelles.composee}
    \begin{question}{derivees.usuelles.composee.7}
          Posons pour tout nombre réel \(x\) strictement positif, \( f(x)=\sqrt{2x} \). Alors \( f'(x)=\ldots\)
         \begin{multicols}{3}     
	    \begin{reponses}
	 	 \mauvaise{\( \dfrac{2}{\sqrt{2x}}   \)}
           	 \mauvaise{\( \dfrac{1}{2\sqrt{2x}}   \)}
		 \mauvaise{\( \dfrac{1}{2\sqrt{x}}   \)}
              \bonne{\( \dfrac{1}{\sqrt{2x}}  \)}
              \mauvaise{\(\dfrac{-1}{2x}\)}
  		 \lastchoices
  		 
  		 \phantom{foo}
	    \end{reponses}
         \end{multicols}
    \end{question}
\end{qcm}
 
\begin{qcm}{derivees.usuelles.composee}
    \begin{question}{derivees.usuelles.composee.8}
          Posons pour tout nombre réel \(x\) non nul, \( f(x)=\dfrac{1}{2x} \). Alors \( f'(x)=\ldots\)
         \begin{multicols}{3}     
	    \begin{reponses}
		  \mauvaise{\(-\dfrac{1}{4x^2}\)}
	 	 \mauvaise{\( -\dfrac{1}{2x}   \)}
         \mauvaise{\( \ln(2 x) \textcolor{white}{\dfrac{1}{2^2}}  \)}
		 \mauvaise{\( \dfrac{1}{x^2}  \)}
  		 \bonne{\( - \dfrac{1}{2 x^2} \)}
%   		 \lastchoices
           \end{reponses}
         \end{multicols}
    \end{question}
\end{qcm}

\begin{qcm}{derivees.usuelles.produit.quotient}
    \begin{question}{derivees.usuelles.produit.quotient.1}
          Posons pour tout nombre réel \(x\) différent de \(-2\), \(f(x)=\dfrac{x+1}{x+2} \). Alors \( f'(x)=\ldots\)
%          \vspace{-1.5ex}
         \begin{multicols}{3}
          \begin{reponses}
		 \mauvaise{\(\dfrac{2x+3}{(x+2)^2}\)}
		 \mauvaise{\( \dfrac{1}{x+2}   \)}
           	 \mauvaise{\( \dfrac{x+1}{(x+2)^2} \)}
		 \mauvaise{\( 1 \textcolor{white}{\dfrac{a}{2^2}} \)}
  		 \bonne{\(  \dfrac{1}{ (x+2)^2} \)}
  		 \lastchoices
  		 
  		 \phantom{foo}
          \end{reponses}
         \end{multicols}
    \end{question}
\end{qcm}

\begin{qcm}{derivees.usuelles.produit.quotient}
    \begin{question}{derivees.usuelles.produit.quotient.2}
         Posons pour tout nombre réel \(x\) différent de \(-2\), \( f(x)=\dfrac{2x}{x+2} \). Alors \( f'(x)=\ldots\)
         \begin{multicols}{3}     
	  \begin{reponses}
		\mauvaise{\(2 \textcolor{white}{\dfrac{2}{2^2}}\)}
	 	 \mauvaise{\( \dfrac{2}{x+2} \textcolor{white}{\dfrac{2}{2^2}}  \)}
           	 \mauvaise{\( \dfrac{-2}{(x+2)^2}   \)}
		 \mauvaise{\( \dfrac{2x}{(x+2)^2}  \)}
        	 \bonne{\(  \dfrac{4}{ (x+2)^2} \)}
  		 \lastchoices
  		 
  		 \phantom{foo}
          \end{reponses}
         \end{multicols}
    \end{question}
\end{qcm}

\begin{qcm}{derivees.usuelles.produit.quotient}
    \begin{question}{derivees.usuelles.produit.quotient.3}
        Posons pour tout nombre réel \(x\), \( f(x)=x {\mathrm{e}}^x \). Alors \( f'(x)=\ldots\)
         \begin{multicols}{3} 
	  \begin{reponses}
	 	 \mauvaise{\( 1 + {\mathrm{e}}^x   \textcolor{white}{\dfrac{1}{e^x}}\)}
           	 \mauvaise{\( x + {\mathrm{e}}^x   \textcolor{white}{\dfrac{1}{e^x}} \)}
		 \mauvaise{\({\mathrm{e}}^x  \textcolor{white}{\dfrac{1}{e^x}} \)}
 		 \mauvaise{\(  \dfrac{1-x}{{\mathrm{e}}^x} \)}
 		 \lastchoices
		 \bonne{\(  {\mathrm{e}}^x(1+x)\)}
  		 
  		 
  		 \phantom{foo}
          \end{reponses}
         \end{multicols}
    \end{question}
\end{qcm}

\begin{qcm}{derivees.usuelles.produit.quotient}
    \begin{question}{derivees.usuelles.produit.quotient.4}
        Posons pour tout nombre réel \(x\) strictement positif, \( f(x)=\ln(x) \ {\mathrm{e}}^x \). Alors \( f'(x)=\ldots\)
         \begin{multicols}{2}         
	  \begin{reponses}
	 	 \mauvaise{\( \ln(x) + {\mathrm{e}}^x  \textcolor{white}{\dfrac{1}{e^x}} \)}
         \mauvaise{\( {\mathrm{e}}^x +\dfrac{1}{x} {\mathrm{e}}^x \textcolor{white}{\dfrac{1}{e^x}}  \)}
		 \mauvaise{\(\dfrac{\frac{1}{x}-\ln(x)}{{\mathrm{e}}^x}  \)}
 		 \mauvaise{\(  \dfrac{1}{x} {\mathrm{e}}^x \textcolor{white}{\dfrac{1}{e^x}}\)}
		 \bonne{\(  {\mathrm{e}}^x\left(\ln(x)+\dfrac{1}{x}\right) \)}
%   		 \lastchoices
%   		 
%   		 \phantom{foo}
          \end{reponses}
         \end{multicols}
    \end{question}
\end{qcm}

\begin{qcm}{derivees.usuelles.produit.quotient}
    \begin{question}{derivees.usuelles.produit.quotient.5}
        Posons pour tout nombre réel \(x\) strictement positif, \( f(x)=\ln(x) \ \sqrt{x} \). Alors \( f'(x)=\ldots\)
         \begin{multicols}{2}         
	  \begin{reponses}
		\mauvaise{\(\dfrac{1}{x}+\dfrac{1}{2\sqrt{x}}\)}
		 \mauvaise{\(\dfrac{1}{2x\sqrt{x}}\)}
		 \mauvaise{\(\dfrac{\ln(x)}{2\sqrt{x}}\)}
        \bonne{\(  \dfrac{1}{\sqrt{x}} \left(1+\dfrac{\ln{x}}{2}\right) \)}

		 \lastchoices
		 \mauvaise{\(  \dfrac{\dfrac{\sqrt{x}}{ x}-\dfrac{\ln{x}}{2 \sqrt{x}}}{x} \)}
%   		 \phantom{foo}
          \end{reponses}
         \end{multicols}
    \end{question}
\end{qcm}

\begin{qcm}{autres.comparaisons}
	\begin{question}{autres.comparaisons.1.n}
		Soit \(0<a<1\). Parmi les 4 nombres réels suivants, cocher celui qui est le plus grand :
		\vspace{-0.2cm}
		\begin{multicols}{2}
			\begin{reponses}
				\bonne{\(\sqrt{a}\)}
				\mauvaise{\(a\)}
				\mauvaise{\(a^2\)}
				\mauvaise{\(a\sqrt{a}\)}
			\end{reponses}
		\end{multicols}
	\end{question}
\end{qcm}

\begin{qcm}{autres.comparaisons}
	\begin{question}{autres.comparaisons.2.n}
		Soit \(0<x<1\). Parmi les 4 nombres réels suivants, cocher celui qui est le plus grand :
		\vspace{-0.2cm}
		\begin{multicols}{2}
			\begin{reponses}
				\bonne{\(  x^{-1}\)}
				\mauvaise{\(x\)}
				\mauvaise{\(x^2\)}
				\mauvaise{\(x\sqrt{x}\)}
			\end{reponses}
		\end{multicols}
	\end{question}
\end{qcm}

\begin{qcm}{autres.comparaisons}
	\begin{question}{autres.comparaisons.3.n}
		Soit \(b<-1\). Parmi les 4 nombres réels suivants, cocher celui qui est le plus grand :
		\vspace{-0.2cm}
		\begin{reponseshoriz}
			\bonne{\( b^2\)}
			\mauvaise{\(b\)}
			\mauvaise{\(b^{-1}\)}
			\mauvaise{\( \left|b\right|\)}
		\end{reponseshoriz}
	\end{question}
\end{qcm}

\begin{qcm}{autres.comparaisons}
	\begin{question}{autres.comparaisons.4.n}
		Soit \(-1<x<0\). Parmi les 4 nombres réels suivants, cocher celui qui est le plus grand~:
		\vspace{-0.2cm}
		\begin{reponseshoriz}
			\mauvaise{\(\dfrac{1}{x}\)}
			\mauvaise{\(x\)}
			\bonne{\(x^2\)}
			\mauvaise{\(x^3\)}
		\end{reponseshoriz}
	\end{question}
\end{qcm}

\begin{qcm}{autres.comparaisons}
	\begin{question}{autres.comparaisons.5.k}
		Soit \(-1<x<0\). Parmi les 4 nombres réels suivants, cocher celui qui est le plus petit~:
		\vspace{-0.2cm}
		\begin{reponseshoriz}
			\bonne{\(\dfrac{1}{x}\)}
			\mauvaise{\(x\)}
			\mauvaise{\(x^2\)}
			\mauvaise{\(x^3\)}
		\end{reponseshoriz}
	\end{question}
\end{qcm}

\begin{qcm}{autres.comparaisons}
	\begin{question}{autres.comparaisons.6.k}
		Soit \(0<x<1\). Parmi les 4 nombres réels suivants, cocher celui qui est le plus grand~:
		\vspace{-0.2cm}
		\begin{reponseshoriz}
			\bonne{\(\dfrac{1}{x}\)}
			\mauvaise{\(\sqrt{x}\)}
			\mauvaise{\(x\)}
			\mauvaise{\(x^2\)}
		\end{reponseshoriz}
	\end{question}
\end{qcm}

\begin{qcm}{autres.comparaisons}
	\begin{question}{autres.comparaisons.7.k}
		Soit \(0<x<1\). Parmi les 4 nombres réels suivants, cocher celui qui est le plus petit~:
		\vspace{-0.2cm}
		\begin{reponseshoriz}
			\mauvaise{\(\dfrac{1}{x}\)}
			\mauvaise{\(\sqrt{x}\)}
			\mauvaise{\(x\)}
			\bonne{\(x^2\)}
		\end{reponseshoriz}
	\end{question}
\end{qcm}



\begin{qcm}{signe.produit.quotient}
    \begin{question}{signe.produit.quotient.1.k}
         Quel est l'ensemble des solutions réelles de \(\dfrac{x}{(x-1)(x+3)}\geqslant 0\)?
         \vspace{0.1cm}
         \begin{reponses}
            \mauvaise{\(]-\infty,-3[\cup]-3,0[\cup]0,1[\cup]1,+\infty[\)}
            \mauvaise{\(]-\infty,-3[\cup[0,1[\)}
            \bonne{\(]-3,0] \cup]1,+\infty[\)}
            \mauvaise{\(]-\infty,-3[\cup]1,+\infty[\)}
            \mauvaise{\([-3,1]\)}
         \end{reponses}
         \vspace{0.4cm}
    \end{question}
\end{qcm}

\begin{qcm}{signe.produit.quotient}
    \begin{question}{signe.produit.quotient.2.k}
         Quel est l'ensemble des solutions réelles de \(\dfrac{x}{(x+1)(x-3)}\geqslant 0\)?
         \vspace{0.1cm}
         \begin{reponses}
            \mauvaise{\(]-\infty,-1[\cup]-1,0[\cup]0,3[\cup]3,+\infty[\)}
            \mauvaise{\(]-\infty,-1[\cup[0,3[\)}
            \bonne{\(]-1,0] \cup]3,+\infty[\)}
            \mauvaise{\(]-\infty,-1[\cup]3,+\infty[\)}
            \mauvaise{\([-1,3]\)}
         \end{reponses}
         \vspace{0.4cm}
    \end{question}
\end{qcm}

\begin{qcm}{signe.produit.quotient}
    \begin{question}{signe.produit.quotient.3.k}
         Quel est l'ensemble des solutions réelles de \(\dfrac{(x+2)x}{x-1}\geqslant 0\)?
         \vspace{0.1cm}
         \begin{reponses}
            \mauvaise{\(]-\infty,-2[\cup]-2,0[\cup]0,1[\cup]1,+\infty[\)}
            \mauvaise{\(]-\infty,-2[\cup]1,+\infty[\)}
            \bonne{\([-2,0] \cup]1,+\infty[\)}
            \mauvaise{\(]-\infty,-2[\cup]0,1[\)}
            \mauvaise{\(]-2,0[\)}
         \end{reponses}
         \vspace{0.4cm}
    \end{question}
\end{qcm}

\begin{qcm}{signe.produit.quotient}
    \begin{question}{signe.produit.quotient.4.k}
         Quel est l'ensemble des solutions réelles de \(\dfrac{x(x-2)}{x+1}\leqslant 0\)?
         \vspace{0.1cm}
         \begin{reponses}
	    \bonne{\(]-\infty,-1[\cup[0,2]\)}
            \mauvaise{\(]-\infty,-1[\cup]-1,0[\cup]0,2[\cup]2,+\infty[\)}
            \mauvaise{\(]-1,2]\)}
            \mauvaise{\(]-1,0[\cup]2,+\infty[\)}
            \mauvaise{\([0,2]\)}
         \end{reponses}
         \vspace{0.4cm}
    \end{question}
\end{qcm}

\begin{qcm}{difficulte.test}
    \begin{questionmult}{difficulte.test.1}
        Sur une échelle de \(1\) à \(5\), \(1\) signifiant "très facile" et \(5\) signifiant "très difficile", comment évalueriez-vous la difficulté de ce test?       
        {\footnotesize \scoring{formula=intX}\QuestionIndicative
        \AMCnumericChoices{0}{digits=1,decimals=0,sign=false,base=6,borderwidth=0pt,backgroundcol=white,approx=0,hspace=1em,vspace=0.5ex,nozero=true,scoring=false}}
    \end{questionmult}
\end{qcm}


%% fabrication des copies


\begin{copieexamen}[2]
\vspace{-2cm}

\fbox{\begin{minipage}[t]{0.5\linewidth}
    \champnom{%
    \begin{minipage}[t]{0.95\linewidth}
        NOM PRÉNOM :\\[0.5cm]
        .\dotfill\\[0.3cm]
        %  Numéro étudiant à coder dans la grille ci-contre :
        \end{minipage}
    }
    \end{minipage}}
\hspace{1cm}
\begin{minipage}[t]{.4\linewidth}
%   \small\AMCcodeH{IDetu}{4}
  \texttt{\small\AMCcodeGrid[h]{IdEtu}{ABCDEFGHI,JKLMNOPQR,STUVWXYZ,1234}}
\end{minipage}


\medskip
\hrule
\medskip
\noindent
Chaque question comporte exactement une bonne réponse. 

% Les mauvaises réponses seront comptées négativement. Merci de ne pas utiliser de crayon de papier ni de stylo effaçable et de bien remplir les cases.
% 
% \noindent En cas d'erreur, recouvrez complètement la case de blanc correcteur et \textbf{ne la redessinez pas}.
\medskip
% \medskip
\hrule
\medskip
% \setlength{\columnseprule}{1pt}

\begin{multicols}{2}
\setlength\columnsep{0cm}

\cleargroup{everything}
\cleargroup{fractions}
\cleargroup{puissances}
\cleargroup{developpement.factorisation}
\cleargroup{derivees}
\cleargroup{exp.ln}
\cleargroup{racine}
\cleargroup{signe}
\cleargroup{trigonometrie}
\cleargroup{fonctions}
\cleargroup{intersection.droites}

% Fractions
\shufflegroup{fractions.numeriques.priorites}\copygroup[1]{fractions.numeriques.priorites}{fractions}
\shufflegroup{fractions.lettres.simple}\copygroup[1]{fractions.lettres.simple}{fractions}
\shufflegroup{fractions.lettres}\copygroup[1]{fractions.lettres}{fractions}

% Puissances
\shufflegroup{puissances.regles}\copygroup[1]{puissances.regles}{puissances}
\shufflegroup{puissances.simplifications}\copygroup[1]{puissances.simplifications}{puissances}
\shufflegroup{puissances.simplifications.2variables}\copygroup[1]{puissances.simplifications.2variables}{puissances}

% Développement et factorisation
\shufflegroup{developpement.factorisation.developpement}\copygroup[1]{developpement.factorisation.developpement}{developpement.factorisation}
\shufflegroup{developpement.factorisation.factorisation}\copygroup[1]{developpement.factorisation.factorisation}{developpement.factorisation}

% Dérivées usuelles et formules de dérivation
\shufflegroup{derivees.usuelles.composee}\copygroup[1]{derivees.usuelles.composee}{derivees}
\shufflegroup{derivees.usuelles.produit.quotient}\copygroup[1]{derivees.usuelles.produit.quotient}{derivees}

% Exponentielle et logarithme
\shufflegroup{expln.simple}\copygroup[1]{expln.simple}{exp.ln}
\shufflegroup{expln.moyen}\copygroup[1]{expln.moyen}{exp.ln}
\shufflegroup{expln.dur}\copygroup[1]{expln.dur}{exp.ln}

% Racines carrées
\shufflegroup{racine.moyen}\copygroup[1]{racine.moyen}{racine}
\shufflegroup{racine.simple}\copygroup[1]{racine.simple}{racine}

% Tableaux de signes
\shufflegroup{signe.polynome}\copygroup[1]{signe.polynome}{signe}
\shufflegroup{signe.produit.quotient}\copygroup[1]{signe.produit.quotient}{signe}

% Trigonométrie
\shufflegroup{trigonometrie.valeurs}\copygroup[1]{trigonometrie.valeurs}{trigonometrie}
\shufflegroup{trigonometrie.lienSinusCosinus}\copygroup[1]{trigonometrie.lienSinusCosinus}{trigonometrie}
\shufflegroup{trigonometrie.symetries}\copygroup[1]{trigonometrie.symetries}{trigonometrie}
\shufflegroup{trigonometrie.encadrements}\copygroup[1]{trigonometrie.encadrements}{trigonometrie}

% Autres
\shufflegroup{autres.variables.muettes}\copygroup[1]{autres.variables.muettes}{fonctions}
\shufflegroup{autres.variations.polynome}\copygroup[1]{autres.variations.polynome}{fonctions}

% Fonctions affines
\shufflegroup{droites}\copygroup[1]{droites}{intersection.droites}
\shufflegroup{droitesbis}\copygroup[1]{droitesbis}{intersection.droites}

% Autres
\shufflegroup{autres.comparaisons}\copygroup[1]{autres.comparaisons}{everything}

% Développement et factorisation
\shufflegroup{developpement.factorisation.puissances}\copygroup[1]{developpement.factorisation.puissances}{everything}


\copygroup[1]{difficulte.test}{everything}

\shufflegroup{fractions}\insertgroup{fractions}
\shufflegroup{puissances}\insertgroup{puissances}
\shufflegroup{developpement.factorisation}\insertgroup{developpement.factorisation}
\shufflegroup{derivees}\insertgroup{derivees}
\shufflegroup{exp.ln}\insertgroup{exp.ln}
\shufflegroup{racine}\insertgroup{racine}
\shufflegroup{signe}\insertgroup{signe}
\shufflegroup{trigonometrie}\insertgroup{trigonometrie}
\shufflegroup{fonctions}\insertgroup{fonctions}
\shufflegroup{intersection.droites}\insertgroup{intersection.droites}
\insertgroup{everything}
\clearpage

\end{multicols}

\end{copieexamen}


\end{document}
