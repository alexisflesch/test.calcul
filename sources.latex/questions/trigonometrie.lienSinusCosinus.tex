\begin{qcm}{trigonometrie.lienSinusCosinus}
    \begin{question}{trigo.lienSinusCosinus.1.k}
         On considère un nombre réel \(x\in\left[2\pi,\frac{5\pi}{2}\right]\) tel que \(\cos(x)=\frac{4}{5}\). Que vaut \(\sin(x)\) ?
         \vspace{-1.5ex}
         \begin{multicols}{4}
         \begin{reponses}
              \mauvaise{\(\dfrac{\sqrt{17}}{5}\)}
              \mauvaise{\(-\dfrac{3}{5}\)}
              \bonne{\(\dfrac{3}{5}\)}
              \mauvaise{\(\dfrac{9}{25}\)}
         \end{reponses}
        \end{multicols}
    \end{question}
\end{qcm}

\begin{qcm}{trigonometrie.lienSinusCosinus}
    \begin{question}{trigo.lienSinusCosinus.2.k}
         On considère un nombre réel \(x\in\left[\frac{3\pi}{2},2\pi\right]\) tel que \(\cos(x)=\frac{4}{5}\). Que vaut \(\sin(x)\) ?
         \vspace{-1.5ex}
         \begin{multicols}{4}
         \begin{reponses}
              \mauvaise{\(\dfrac{\sqrt{17}}{5}\)}
              \bonne{\(-\dfrac{3}{5}\)}
              \mauvaise{\(\dfrac{3}{5}\)}
              \mauvaise{\(\dfrac{9}{25}\)}
         \end{reponses}
        \end{multicols}
    \end{question}
\end{qcm}

\begin{qcm}{trigonometrie.lienSinusCosinus}
    \begin{question}{trigo.lienSinusCosinus.3.k}
         On considère un nombre réel \(x\in\left[3\pi,\frac{7\pi}{2}\right]\) tel que \(\cos(x)=-\frac{4}{5}\). Que vaut \(\sin(x)\) ?
         \vspace{-1.5ex}
         \begin{multicols}{4}
         \begin{reponses}
              \mauvaise{\(\dfrac{\sqrt{17}}{5}\)}
              \bonne{\(-\dfrac{3}{5}\)}
              \mauvaise{\(\dfrac{3}{5}\)}
              \mauvaise{\(\dfrac{9}{25}\)}
         \end{reponses}
        \end{multicols}
    \end{question}
\end{qcm}

\begin{qcm}{trigonometrie.lienSinusCosinus}
    \begin{question}{trigo.lienSinusCosinus.4.k}
         On considère un nombre réel \(x\in\left[\frac{5\pi}{2},3\pi\right]\) tel que \(\cos(x)=-\frac{4}{5}\). Que vaut \(\sin(x)\) ?
         \vspace{-1.5ex}
         \begin{multicols}{4}
         \begin{reponses}
              \mauvaise{\(\dfrac{\sqrt{17}}{5}\)}
              \mauvaise{\(-\dfrac{3}{5}\)}
              \bonne{\(\dfrac{3}{5}\)}
              \mauvaise{\(\dfrac{9}{25}\)}
         \end{reponses}
        \end{multicols}
    \end{question}
\end{qcm}
