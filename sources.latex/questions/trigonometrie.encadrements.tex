\begin{qcm}{trigonometrie.encadrements}
    \begin{question}{trigonometrie.encadrements.1.k}
         Soit \(0<x<\frac{\pi}{8}\). Encadrer \(\frac{1}{\cos^2(2x)}\) par deux entiers consécutifs : 
         \( n<\frac{1}{\cos^2(2x)}<n+1\) où \(n=\ldots\)
         \vspace{-0.1cm}
         \begin{reponseshoriz}
            \mauvaise{\(-1\)}
            \mauvaise{\(0\)}
            \bonne{\(1\)}
            \mauvaise{\(2\)}
         \end{reponseshoriz}
    \end{question}
\end{qcm}

\begin{qcm}{trigonometrie.encadrements}
    \begin{question}{trigonometrie.encadrements.2.k}
         Soit \(0<x<\frac{\pi}{4}\). Encadrer \(\frac{1}{\cos^2(x)}\) par deux entiers consécutifs : 
         \( n<\frac{1}{\cos^2(x)}<n+1\) où \(n=\ldots\)
         \vspace{-0.1cm}
         \begin{reponseshoriz}
            \mauvaise{\(-1\)}
            \mauvaise{\(0\)}
            \bonne{\(1\)}
            \mauvaise{\(2\)}
         \end{reponseshoriz}
    \end{question}
\end{qcm}

\begin{qcm}{trigonometrie.encadrements}
    \begin{question}{trigonometrie.encadrements.3.k}
         Soit \(\frac{\pi}{4}<x<\frac{\pi}{2}\). Encadrer \(\frac{1}{\sin^2(x)}\) par deux entiers consécutifs : 
         \( n<\frac{1}{\sin^2(x)}<n+1\) où \(n=\ldots\)
         \vspace{-0.1cm}
         \begin{reponseshoriz}
            \mauvaise{\(-1\)}
            \mauvaise{\(0\)}
            \bonne{\(1\)}
            \mauvaise{\(2\)}
         \end{reponseshoriz}
    \end{question}
\end{qcm}

\begin{qcm}{trigonometrie.encadrements}
    \begin{question}{trigonometrie.encadrements.3.k}
         Soit \(\frac{\pi}{8}<x<\frac{\pi}{4}\). Encadrer \(\frac{1}{\sin^2(2x)}\) par deux entiers consécutifs : 
         \( n<\frac{1}{\sin^2(2x)}<n+1\) où \(n=\ldots\)
         \vspace{-0.3cm}
         \begin{reponseshoriz}
            \mauvaise{\(-1\)}
            \mauvaise{\(0\)}
            \bonne{\(1\)}
            \mauvaise{\(2\)}
         \end{reponseshoriz}
    \end{question}
\end{qcm}
