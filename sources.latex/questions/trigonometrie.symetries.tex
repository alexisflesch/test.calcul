\begin{qcm}{trigonometrie.symetries}
    \begin{question}{trigo.symetries.1.k}
         Soit \(x\) un nombre réel. \(\sin\left(\pi-x\right)=\ldots\)
         \vspace{-1.5ex}
         \begin{multicols}{2}
         \begin{reponses}
              \mauvaise{\(\cos(x)\)}
              \mauvaise{\(-\cos(x)\)}
              \bonne{\(\sin(x)\)}
              \mauvaise{\(-\sin(x)\)}
         \end{reponses}
        \end{multicols}
    \end{question}
\end{qcm}

\begin{qcm}{trigonometrie.symetries}
    \begin{question}{trigo.symetries.2.k}
         Soit \(x\) un nombre réel. \(\cos\left(\pi-x\right)=\ldots\)
         \vspace{-1.5ex}
         \begin{multicols}{2}
         \begin{reponses}
              \mauvaise{\(\cos(x)\)}
              \bonne{\(-\cos(x)\)}
              \mauvaise{\(\sin(x)\)}
              \mauvaise{\(-\sin(x)\)}
         \end{reponses}
        \end{multicols}
    \end{question}
\end{qcm}

\begin{qcm}{trigonometrie.symetries}
    \begin{question}{trigo.symetries.3.k}
         Soit \(x\) un nombre réel. \(\sin\left(x+\frac{\pi}{2}\right)=\ldots\)
         \vspace{-1.5ex}
         \begin{multicols}{2}
         \begin{reponses}
              \mauvaise{\(\cos(x)\)}
              \mauvaise{\(-\cos(x)\)}
              \mauvaise{\(\sin(x)\)}
              \bonne{\(-\sin(x)\)}
         \end{reponses}
        \end{multicols}
    \end{question}
\end{qcm}

\begin{qcm}{trigonometrie.symetries}
    \begin{question}{trigo.symetries.4.k}
         Soit \(x\) un nombre réel. \(\cos\left(x+\frac{\pi}{2}\right)=\ldots\)
         \vspace{-1.5ex}
         \begin{multicols}{2}
         \begin{reponses}
              \mauvaise{\(\cos(x)\)}
              \mauvaise{\(-\cos(x)\)}
              \mauvaise{\(\sin(x)\)}
              \bonne{\(-\sin(x)\)}
         \end{reponses}
        \end{multicols}
    \end{question}
\end{qcm}
