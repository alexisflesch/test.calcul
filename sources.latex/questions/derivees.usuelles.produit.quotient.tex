\begin{qcm}{derivees.usuelles.produit.quotient}
    \begin{question}{derivees.usuelles.produit.quotient.1}
          Posons pour tout nombre réel \(x\) différent de \(-2\), \(f(x)=\dfrac{x+1}{x+2} \). Alors \( f'(x)=\ldots\)
%          \vspace{-1.5ex}
         \begin{multicols}{3}
          \begin{reponses}
		 \mauvaise{\(\dfrac{2x+3}{(x+2)^2}\)}
		 \mauvaise{\( \dfrac{1}{x+2}   \)}
           	 \mauvaise{\( \dfrac{x+1}{(x+2)^2} \)}
		 \mauvaise{\( 1 \textcolor{white}{\dfrac{a}{2^2}} \)}
  		 \bonne{\(  \dfrac{1}{ (x+2)^2} \)}
  		 \lastchoices
  		 
  		 \phantom{foo}
          \end{reponses}
         \end{multicols}
    \end{question}
\end{qcm}

\begin{qcm}{derivees.usuelles.produit.quotient}
    \begin{question}{derivees.usuelles.produit.quotient.2}
         Posons pour tout nombre réel \(x\) différent de \(-2\), \( f(x)=\dfrac{2x}{x+2} \). Alors \( f'(x)=\ldots\)
         \begin{multicols}{3}     
	  \begin{reponses}
		\mauvaise{\(2 \textcolor{white}{\dfrac{2}{2^2}}\)}
	 	 \mauvaise{\( \dfrac{2}{x+2} \textcolor{white}{\dfrac{2}{2^2}}  \)}
           	 \mauvaise{\( \dfrac{-2}{(x+2)^2}   \)}
		 \mauvaise{\( \dfrac{2x}{(x+2)^2}  \)}
        	 \bonne{\(  \dfrac{4}{ (x+2)^2} \)}
  		 \lastchoices
  		 
  		 \phantom{foo}
          \end{reponses}
         \end{multicols}
    \end{question}
\end{qcm}

\begin{qcm}{derivees.usuelles.produit.quotient}
    \begin{question}{derivees.usuelles.produit.quotient.3}
        Posons pour tout nombre réel \(x\), \( f(x)=x {\mathrm{e}}^x \). Alors \( f'(x)=\ldots\)
         \begin{multicols}{3} 
	  \begin{reponses}
	 	 \mauvaise{\( 1 + {\mathrm{e}}^x   \textcolor{white}{\dfrac{1}{e^x}}\)}
           	 \mauvaise{\( x + {\mathrm{e}}^x   \textcolor{white}{\dfrac{1}{e^x}} \)}
		 \mauvaise{\({\mathrm{e}}^x  \textcolor{white}{\dfrac{1}{e^x}} \)}
 		 \mauvaise{\(  \dfrac{1-x}{{\mathrm{e}}^x} \)}
 		 \lastchoices
		 \bonne{\(  {\mathrm{e}}^x(1+x)\)}
  		 
  		 
  		 \phantom{foo}
          \end{reponses}
         \end{multicols}
    \end{question}
\end{qcm}

\begin{qcm}{derivees.usuelles.produit.quotient}
    \begin{question}{derivees.usuelles.produit.quotient.4}
        Posons pour tout nombre réel \(x\) strictement positif, \( f(x)=\ln(x) \ {\mathrm{e}}^x \). Alors \( f'(x)=\ldots\)
         \begin{multicols}{2}         
	  \begin{reponses}
	 	 \mauvaise{\( \ln(x) + {\mathrm{e}}^x  \textcolor{white}{\dfrac{1}{e^x}} \)}
         \mauvaise{\( {\mathrm{e}}^x +\dfrac{1}{x} {\mathrm{e}}^x \textcolor{white}{\dfrac{1}{e^x}}  \)}
		 \mauvaise{\(\dfrac{\frac{1}{x}-\ln(x)}{{\mathrm{e}}^x}  \)}
 		 \mauvaise{\(  \dfrac{1}{x} {\mathrm{e}}^x \textcolor{white}{\dfrac{1}{e^x}}\)}
		 \bonne{\(  {\mathrm{e}}^x\left(\ln(x)+\dfrac{1}{x}\right) \)}
%   		 \lastchoices
%   		 
%   		 \phantom{foo}
          \end{reponses}
         \end{multicols}
    \end{question}
\end{qcm}

\begin{qcm}{derivees.usuelles.produit.quotient}
    \begin{question}{derivees.usuelles.produit.quotient.5}
        Posons pour tout nombre réel \(x\) strictement positif, \( f(x)=\ln(x) \ \sqrt{x} \). Alors \( f'(x)=\ldots\)
         \begin{multicols}{2}         
	  \begin{reponses}
		\mauvaise{\(\dfrac{1}{x}+\dfrac{1}{2\sqrt{x}}\)}
		 \mauvaise{\(\dfrac{1}{2x\sqrt{x}}\)}
		 \mauvaise{\(\dfrac{\ln(x)}{2\sqrt{x}}\)}
        \bonne{\(  \dfrac{1}{\sqrt{x}} \left(1+\dfrac{\ln{x}}{2}\right) \)}

		 \lastchoices
		 \mauvaise{\(  \dfrac{\dfrac{\sqrt{x}}{ x}-\dfrac{\ln{x}}{2 \sqrt{x}}}{x} \)}
%   		 \phantom{foo}
          \end{reponses}
         \end{multicols}
    \end{question}
\end{qcm}