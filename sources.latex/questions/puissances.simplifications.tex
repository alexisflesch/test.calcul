\begin{qcm}{puissances.simplifications}
    \begin{questionmult}{puissances.simplifications.1.k}
	Si \(x>0\), alors \(\dfrac{x^2\times x^{-6} \times x^9}{x^{-3}\times {\sqrt{x^{10}}}}=x^p\) où \(p=\ldots\)
	
	\noindent\textit{Cocher ci-dessous le signe de \(p\) et sa valeur.}
	\vspace{-1.2ex}
	\footnotesize{
	 \AMCnumericChoices{3}{digits=1,decimals=0,sign=true,borderwidth=0pt,backgroundcol=white,approx=0,hspace=0.4em,vspace=0.5ex}}
    \end{questionmult}
\end{qcm}

\begin{qcm}{puissances.simplifications}
    \begin{questionmult}{puissances.simplifications.2.k}
	Si \(x \neq 0\), alors \(\dfrac{x^{-3}\times x^5}{{\sqrt{x^4}}\times x^{-6} \times x^9}=x^p\) où \(p=\ldots\)
    
    \noindent\textit{Cocher ci-dessous le signe de \(p\) et sa valeur.}
    \vspace{-1.2ex}
    \footnotesize{
    \AMCnumericChoices{-3}{digits=1,decimals=0,sign=true,borderwidth=0pt,backgroundcol=white,approx=0,hspace=0.4em,vspace=0.5ex}}
    \end{questionmult}
\end{qcm}

\begin{qcm}{puissances.simplifications}
    \begin{questionmult}{puissances.simplifications.3.k}
	Si \(x \neq 0\), alors \(\dfrac{x^7\times x^{-2} \times {\sqrt{x^6}}}{x^{-4}\times x^8 \times x^5}=x^p\) où \(p=\ldots\)
    
    \noindent\textit{Cocher ci-dessous le signe de \(p\) et sa valeur.}
    \vspace{-1.2ex}
    \footnotesize{
    	  \AMCnumericChoices{-1}{digits=1,decimals=0,sign=true,borderwidth=0pt,backgroundcol=white,approx=0,hspace=0.4em,vspace=0.5ex}}
    \end{questionmult}
\end{qcm}

\begin{qcm}{puissances.simplifications}
    \begin{questionmult}{puissances.simplifications.4.k}
	Si \(x \neq 0\), alors \(\dfrac{x^{-4}\times x^8 \times {\sqrt{x^{10}}}}{x^7\times x^{-2} \times x^3}=x^p\) où \(p=\ldots\)
    
    \noindent\textit{Cocher ci-dessous le signe de \(p\) et sa valeur.}
    \vspace{-1.2ex}
    \footnotesize{
    	  \AMCnumericChoices{1}{digits=1,decimals=0,sign=true,borderwidth=0pt,backgroundcol=white,approx=0,hspace=0.4em,vspace=0.5ex}}
    \end{questionmult}
\end{qcm}

\begin{qcm}{puissances.simplifications}
    \begin{questionmult}{puissances.simplifications.5.k}
	Si \(x \neq 0\), alors \(\dfrac{x^{-8}\times {\sqrt{x^6}}}{x^{-4}\times x^{10} \times x^{-2}}=x^p\) où \(p=\ldots\)
	
    \noindent\textit{Cocher ci-dessous le signe de \(p\) et sa valeur.}
    \vspace{-1.2ex}
    \footnotesize{
    	  \AMCnumericChoices{-9}{digits=1,decimals=0,sign=true,borderwidth=0pt,backgroundcol=white,approx=0,hspace=0.4em,vspace=0.5ex}}
    \end{questionmult}
\end{qcm}

\begin{qcm}{puissances.simplifications}
    \begin{questionmult}{puissances.simplifications.6.k}
	Si \(x \neq 0\), alors \(\dfrac{x^{-4}\times x^{10} \times x^{-2}}{x^{-8}\times {\sqrt{x^6}}}=x^p\) où \(p=\ldots\)
    
    \noindent\textit{Cocher ci-dessous le signe de \(p\) et sa valeur.}
    \vspace{-1.2ex}
    \footnotesize{
    	  \AMCnumericChoices{9}{digits=1,decimals=0,sign=true,borderwidth=0pt,backgroundcol=white,approx=0,hspace=0.4em,vspace=0.5ex}}
    \end{questionmult}
\end{qcm}


  
  

