\begin{qcm}{fractions.lettres.simple}
    \begin{question}{fractions.lettres.simple.1.l}
         Soit \(x\) un réel non nul. Alors \( \dfrac{1}{x}+\dfrac{x}{2}=\ldots\)
%          \vspace{-1.5ex}
         \begin{multicols}{2}
            \begin{reponses}
              \mauvaise{\( \dfrac{1}{x}+\dfrac{1}{2} \)}
              \mauvaise{\( \dfrac{2+x}{x} \)}
              \mauvaise{\( \dfrac{2+x}{2x} \)}
              \bonne{\( \dfrac{2+x^2}{2x} \)}
         \end{reponses}
         \end{multicols}
    \end{question}
\end{qcm}


\begin{qcm}{fractions.lettres.simple}
    \begin{question}{fractions.lettres.simple.2.a}
         Soit \(x\) un réel non nul. Alors \( \dfrac{3}{x}+\dfrac{x}{2}=\ldots\)
%          \vspace{-1.5ex}
         \begin{multicols}{2}
            \begin{reponses}
              \mauvaise{\( \dfrac{3}{x}+\dfrac{1}{2} \)}
              \mauvaise{\( \dfrac{3+x}{2x} \)}
              \mauvaise{\( \dfrac{3+x}{x+2} \)}
              \bonne{\( \dfrac{6+x^2}{2x} \)}
         \end{reponses}
         \end{multicols}
    \end{question}
\end{qcm}



\begin{qcm}{fractions.lettres}
\begin{question}{fractions.lettres.2.l}
      Soit \(x\) un réel non nul. Simplifier : \( \dfrac{\dfrac{3}{2x}}{~~x^2~~}\).
%          \vspace{-1.5ex}
         \begin{multicols}{3}
            \begin{reponses}
              \mauvaise{\( \dfrac{3x}{2} \)}
              \mauvaise{\( \dfrac{3}{2x} \)}
              \bonne{\( \dfrac{3}{2x^3} \)}
              \mauvaise{\(\dfrac{6}{x}\)}
              \mauvaise{\(\dfrac{2}{3x}\)}
         \end{reponses}
         \end{multicols}
    \end{question}
\end{qcm}

\begin{qcm}{fractions.lettres}
\begin{question}{fractions.lettres.3.l}
         Soit \(x\) un réel non nul. Simplifier : \( \dfrac{3x}{~~\dfrac{2}{x^2}~~}\).
%          \vspace{-1.5ex}
         \begin{multicols}{3}
            \begin{reponses}
              \mauvaise{\( \dfrac{3}{2x} \)}
              \mauvaise{\( \dfrac{3}{2x^3} \)}
              \bonne{\( \dfrac{3x^3}{2} \)}
              \mauvaise{\(\dfrac{2}{3x^3}\)}
              \mauvaise{\( \dfrac{6}{x} \)}
         \end{reponses}
         \setcounter{unbalance}{1}
         \end{multicols}
    \end{question}
\end{qcm}

\begin{qcm}{fractions.lettres}
\begin{question}{fractions.lettres.4.l}
         Soit \(x\) un réel non nul. Simplifier : \( \dfrac{3x}{~~\dfrac{x^2}{2}~~}\).
%          \vspace{-1.5ex}
         \begin{multicols}{3}
            \begin{reponses}
              \mauvaise{\( \dfrac{3}{2x} \)}
             \mauvaise{\( \dfrac{3x}{2} \)}
             \mauvaise{\( \dfrac{3}{2x^2} \)}
              \mauvaise{\( \dfrac{3}{2x^3} \)}
              \bonne{\( \dfrac{6}{x} \)}
         \end{reponses}
         \setcounter{unbalance}{1}
         \end{multicols}
    \end{question}
\end{qcm}


\begin{qcm}{fractions.lettres}
\begin{question}{fractions.lettres.5.l}
         Soit \(x\) un réel non nul. Simplifier : \( \dfrac{\dfrac{3}{2x}}{~~\dfrac{2}{x^2}~~}\).
%          \vspace{-1.5ex}
         \begin{multicols}{3}
            \begin{reponses}
              \mauvaise{\( \dfrac{3}{2x} \)}
              \mauvaise{\( \dfrac{3x}{2} \)}
              \mauvaise{\( \dfrac{3}{2x^2} \)}
              \mauvaise{\( \dfrac{3}{x^3} \)}
              \bonne{\( \dfrac{3x}{4} \)}
         \end{reponses}
         \setcounter{unbalance}{1}
         \end{multicols}
    \end{question}
\end{qcm}